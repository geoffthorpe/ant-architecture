\documentstyle[11pt,makeidx,psfig]{book}
\begin{document}

\chapter{Exceptions and Exception Handling}

Ant-32 instructions generate exceptions when errors occur.
Here we describe how to deal with exceptions using 
exception handlers. And exception handler is a sequence
of instructions that is executed when an exception is 
generated. 

When an exception occurs:

\begin{itemize}

\item The exception registers {\tt e0-e3} contain the exception
        context, as described above.  In particular, {\tt e3} contains
        information about the type of exception that occurred.  {\tt
        e3} can be interpreted as shown in Figures \ref{e3-table} and
        \ref{exceptions-table}.

        Note that at most one of bits 1-3 of {\tt e3} will be set.
        None of the three bits will be set if the exception was due to
        anything other than a memory access.

\item The interrupt disable flag and the exception disable flag are
        set.

\item Execution continues at the exception handler address.

\end{itemize}

By providing the address and code for an exception
handler and the sequence of instructions, you can handle 
exceptions generated by Ant-32.

[some more information, maybe, on what is stored in the exception
 registers]

The following instructions are needed for exception handling in 
Ant-32: 

\begin{verbatim}
        cle        # enable exceptions
        leh des    # load the address of the exception handler from R(des) 

[some more information, maybe, on what is stored in the exception
 registers]
\end{verbatim}

\section{Writing an Exception Handler}

The following are the basic steps required to write an exception handler

\begin{enumerate}

\item Write the instructions that must be executed when the
  exception is generated. 
\item Load the address of the exception handler so that Ant 
  knows where it is when the time comes to execute it (this is 
  the address at which the instructions in step 1 begin.
\item Enable exceptions. When an exception occurs, control
  is then transferred to the address you specified in step 2.

\end{enumerate}

The following code fragment shows how to write an exception handler
and load it so that it is executed when an exception occurs:

{\small
\begin{verbatim}
        lc      r40, 0x8000002c    # Load the address of the 
                                   # exception handler into r40
        leh     r40                # Load the exception handler 
                                   # from the address in r40 
        cle                        # Enable exceptions

        lc r49, 0xdeadbeef         # This is the exception handler. It 
                                   # doesn't do anything meaningful
        halt                       # just save a value and exit
\end{verbatim}}

\end{document}

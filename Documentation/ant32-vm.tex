% $Id: ant32-vm.tex,v 1.3 2002/04/17 20:05:58 ellard Exp $

\chapter{Virtual Memory Architecture}
\label{VirtualMemoryArchitecture}

Ant-32 is a paged architecture, with additional segmentation in the
virtual address space.  The page size is 4KB.

Ant-32 contains a software-managed translation look-aside buffer (TLB)
that maps virtual addresses to physical addresses.  The TLB contains
at least 16 entries, and the number of entries must be a power of two. 
A TLB miss generates an exception.

The top 2 bits of each virtual address determine the segment number.
Segment 0 is the only segment accessible in user mode, while
segments 1-3 are accessible only in supervisor mode.

Ant-32 supports up to one GB of physical address space.  Physical
memory begins at address 0, but need not be contiguous.  Memory-mapped
devices are typically located at the highest addresses, but the
implementor is free to place them wherever necessary.

Each segment is one GB in size, corresponding to the size of physical
memory, but corresponds to a different way of interpreting virtual
addresses or accessing the memory.  Segments 0 and 1 are mapped
through the TLB, and may be cached.  Segments 2 and 3 are not mapped
through the TLB-- the physical address for each virtual address in
these segments is formed by removing the top two bits from the virtual
address.  Segment 2 may be cached, but segment 3 is never cached for
either read or write access (and is intended to be used for
memory-mapped devices).

The next 18 bits of each virtual address are the virtual page number. 
The bottom 12 bits are the offset into the page.

\begin{figure}[ht]
\caption{TLB Entry Format}

\begin{center}
\begin{tabular}{l|p{0.7in}|p{2.5in}|p{1.2in}|}
\cline{2-4}
		& bits 31-30 & bits 29-12	& bits 11-0 \\
\cline{2-4}
Upper Word	& 0	& Physical page	number	& Page Attributes \\
\cline{2-4}
Lower Word	& Segment & Virtual page number	& {\em (Available for OS)} \\
\cline{2-4}
\end{tabular}
\end{center}
\end{figure}

Each TLB entry consists of two 32-bit words.  The top 20 bits of the
upper word are the top 20 bits of the physical address of the page (if
the VALID bit is not set in the lower word).  Note that since Ant-32
physical addresses are only 30 bits long, the upper two bits of the
address, when written as a 32-bit quantity, must always be zero.  The
lower 12 bits of the lower word contain the page attributes bits.  The
page attributes include {\sc VALID}, {\sc READ}, {\sc WRITE}, {\sc
EXEC}, {\sc DIRTY}, and {\sc UNCACHE} bits, as defined in figure
\ref{TLB-attr-bits}.  The remaining bits are reserved.

The top 20 bits of the lower word are the top 20 bits of the virtual
address (the segment number and the virtual page number for that
address).  The lower 12 bits of the upper word are ignored by the
address translation logic, but are available to be used by the
operating system to hold relevant information about the page.

\begin{figure}[ht]
\caption{\label{TLB-attr-bits} TLB Page Attribute Bits}

\begin{center}
\begin{tabular}{|p{1.0in}|p{0.2in}|p{0.2in}|p{0.2in}|p{0.2in}|p{0.2in}|p{0.2in}|}
\hline
	{\em Reserved} & U & D & V & R & W & X \\
\hline
\end{tabular}
\end{center}

\begin{center}
\begin{tabular}{|l|l|p{4in}|}
\hline
{\bf Name}	& {\bf Bit}	& {\bf Description} \\
\hline
\hline
{\sc EXEC}	& 0	& Instruction fetch memory access to addresses mapped
				by this TLB entry is allowed. \\
\hline
{\sc WRITE}	& 1 	& Write memory access to addresses mapped
				by this TLB entry is allowed. \\
\hline
{\sc READ}	& 2	& Read memory access to addresses mapped
				by this TLB entry is allowed.  \\
\hline
{\sc VALID}	& 3	& Indicates a valid TLB entry.  When this is
			set to 0, the contents of the rest of the TLB
			entry are irrelevant.  \\

\hline
{\sc DIRTY}	& 4	& Indicates a dirty page.  When this is set to 1,
			it indicates that the page referenced by this TLB
			entry has been written.  This bit is set to 1
			automatically whenever a write occurs to the page,
			but can be reset to 0 using the instructions that
			modify TLB entries. \\

\hline
{\sc UNCACHE}	& 5	& An uncacheable page.  When this is set to 1,
			the page referenced by the entry will not be
			cached in any processor cache.  \\

\hline
\end{tabular}
\end{center}
\end{figure}

Note that the top bit of the virtual address in the TLB will always be
zero, because only segments 0 and 1 are mapped through the TLB, and
the top two bits of the physical address will always be zero because
physical addresses have only 30 bits.  If values other than zero are
assigned to these bits the result is undefined.

Translation from virtual to physical addresses is done as follows for
any fetch, load, or store:

\begin{enumerate}

	\item The virtual address is split into the segment, virtual
		page number, and page offset.

	\item If the page offset is not divisible by the size of the
		data being fetched, loaded, or stored, an alignment
		exception occurs.

		All memory accesses must be aligned according to their
		size.  In Ant-32, there are only two sizes-- bytes, and
		4-byte words.  Word addresses must be divisible by 4,
		while byte addresses are not restricted.

	\item If the segment is not 0 and the CPU is in user mode, a
		segment privilege exception occurs.

	\item If the segment is 2 or 3, then the virtual address (with
		the segment bits set to zero) is treated as the
		physical address, and the algorithm terminates.

	\item The TLB is searched for an entry corresponding to the
		segment and virtual page number, and with its {\sc
		VALID} bit set to 1.  If no such entry exists, a TLB
		miss exception occurs.

		Note that if there are two or more valid TLB entries
		corresponding to the same virtual page, exception 13
		(TLB multiple match) will occur when the entry table
		is searched.  (This exception will also occur if the
		{\tt tlbpi} instruction is used to search the TLB
		table.)

	\item If the operation is not permitted by the page, a
		TLB protection exception occurs.

		Note that a memory location can be fetched for
		execution if its TLB entry is marked as executable
		even if it is not marked as readable.

	\item Otherwise, the physical address is constructed from the
		top 20 bits of the upper word of the TLB and the lower
		12 bits of the virtual address.

	\item If the physical address does not exist (which can only
		be detected when a memory operation is performed) a
		bus error exception occurs.

\end{enumerate}

Any of the exceptions that can occur during this process can be caught
by the operating system, and the offending instruction can be
restarted if appropriate.

Note that the order in which the conditions that trigger these
exceptions are checked is {\bf undefined}, except for the dependency
that segment privileges are checked before the TLB is probed, and the
TLB protection is checked before the address is sent to the memory
system.  For example, attempting to load a word from address {\tt
0x80001001} in user mode can cause either an alignment exception
(because the page offset is not divisible by 4) or a segment privilege
exception (because the segment is 1 and the processor is in user
mode).

Along with a 30-bit physical address, an {\em uncache} bit is sent to
the memory system whenever a memory operation is performed.  This bit
is defined by the following rules:

\begin{itemize}

\item If the segment is 0 or 1, then the {\em uncache} bit is set to
	the value of the {\sc UNCACHE} bit of the TLB entry that maps
	that address.

\item If the segment is 2, then the {\em uncache} bit is set to 0.

\item If the segment is 3, then the {\em uncache} bit is set to 1.

\end{itemize}

If the implementation does not include a cache, or the cache
is disabled, then this bit is ignored.



% 02/13/94
% $Id: ant.tex,v 1.6 2002/03/22 16:23:38 ellard Exp $

\section{{\sc Ant-8} \input{../CurrVersion} Architecture Overview}

The Ant-8 architecture is a load/store architecture;
the only instructions that can access memory
are the {\em load} and {\em store} (and in some sense
the {\em sys}) instructions.
All other operations access only registers.

The Ant-8 CPU has 16 registers, named {\tt r0} through {\tt r15}.
Register {\tt r0} always contains the constant 0, and register
{\tt r1} is used to hold results related to previous operations
(described later).  {\tt r0} and {\tt r1} are {\em read-only}
and cannot be used as destination registers.
The other 14 registers ({\tt r2} through {\tt r15}) are
general-purpose registers.

\section{General Instructions}
\label{mnemonic-sec}

In the description of the instructions, the notation described in
figure \ref{operand-table} is used.

\begin{figure}
\caption{ \label{operand-table} Assembly Language Operand Types}
\vspace{3mm}
\begin{center}
\begin{tabular}{|lp{4.5in}|}
\hline
{\em des}       & Must always be a register, but never {\tt r0} or {\tt r1}. \\
{\em reg}       & Must always be a register. \\
{\em src1}      & Must always be a register. \\
{\em src2}      & Must always be a register. \\
{\em const8}     & Must be an 8-bit constant (-128 .. 127):
			an integer (signed), char, or label. \\
{\em uconst8}	& Must be an 8-bit constant (0 .. 255):
			an integer (unsigned) or label. \\
{\em uconst4}	& Must be a 4-bit constant integer (0 .. 15). \\
\hline
\end{tabular}
\end{center}
\vspace{3mm}
\end{figure}

The Ant-8 assembly language instructions are listed in figure
\ref{mnemonic-table}.

Note that for all instructions except {\tt sys}, register {\tt r1} is
always updated {\em after} the rest of the instruction is done, so
that it is always safe to use {\tt r1} as a source register for these
instructions.  The {\tt sys} instruction, however, always sets {\tt r1} to 0
before executing the system call.

\begin{figure}
\caption{ \label{mnemonic-table} Ant-8 Instruction Mnemonics }
\vspace{3mm}
\noindent
\begin{tabular}{|ll|p{4.5in}|}
\hline
        {\bf Op}        & {\bf Operands}        & {\bf Description}     \\
\hline
\hline
        {\tt add}       & {\em des, src1, src2} &
                {\em des} gets {\em src1} + {\em src2}.
                {\tt r1} = 1 if the result is $>$ 127, -1 if $<$ -128, or
		0 otherwise. \\
\hline
        {\tt sub}       & {\em des, src1, src2} &
                {\em des} gets {\em src1} - {\em src2}.
                {\tt r1} = 1 if the result is $>$ 127, -1 if $<$ -128, or
		0 otherwise. \\
\hline
        {\tt mul}       & {\em des, src1, src2} &
                Multiply {\em src1} and {\em src2},
                leaving the low-order byte in register {\em des}
                and the high-order byte in register {\tt r1}. \\
\hline
        {\tt div}       & {\em des, src1, src2} &
                Divide {\em src1} by {\em src2},
                leaving the quotient in register {\em des}
                and the remainder in register {\tt r1}. \\
\hline
        {\tt and}       & {\em des, src1, src2} &
                {\em des} gets the bitwise logical {\sc and} of
                {\em src1} and {\em src2}.  {\tt r1} gets the
                bitwise negation of the {\sc and} of {\em src1} and {\em src2}. \\
\hline
        {\tt or}        & {\em des, src1, src2} &
                {\em des} gets the bitwise logical {\sc or} of
                {\em src1} and {\em src2}.  {\tt r1} gets the
                bitwise negation of the {\sc or} of {\em src1} and {\em src2}. \\
\hline
        {\tt shf}        & {\em des, src1, src2} &
		{\em des} gets the bitwise shift of {\em src1} by
		{\em src2} positions.  If {\em src2} is positive,
		{\em src1} is shifted to the left, if {\em src2}
		is negative {\em src1} is shifted to the right. \\
\hline
{\tt beq}       & {\em reg, src1, src2} &
		Branch to {\em reg} if {\em src1} == {\em src2}.
		{\tt r1} is set to the address of the instruction
		following the {\tt beq}. \\
\hline
{\tt bgt}       & {\em reg, src1, src2} &
		Branch to {\em reg} if {\em src1} $>$ {\em src2}.
		{\tt r1} is set to the address of the instruction
		following the {\tt bgt}. \\
\hline
{\tt ld1}        & {\em des, src1, uconst4} & 
		Load the byte at {\em src1 + uconst4} into {\em des}.
		{\tt r1} is unchanged.  \\
\hline
{\tt st1}        & {\em reg, src1, uconst4} &
		Store the contents of register
		{\em reg} to {\em src1 + uconst4}.
		{\tt r1} is unchanged.  \\
\hline
{\tt lc}        & {\em des, const8}      & 
		Load the constant {\em const8} into {\em des}.
		{\tt r1} is unchanged. \\
\hline
{\tt jmp}	& {\em uconst8}	&
		Branch unconditionally to the specified constant.
		{\tt r1} is set to the address
		of the instruction following the {\tt jmp}. \\
\hline
{\tt inc}	& {\em reg, const8}	&
		Add {\em const8} to the specified register.
                {\tt r1} = 1 if the result is $>$ 127, -1 if $<$ -128, or
		0 otherwise. \\
\hline
        {\tt sys}   & {\em reg, code}       &
                Makes a system call.
                See figure \ref{syscall-table}
		for a list of the Ant-8 system calls.
        \\
\hline 
\end{tabular}
\end{figure}
\vspace{3mm}

\subsection{Ant-8 System Calls}
\label{syscall-sec}

All syscalls set {\tt r1} to 0 if successful, and set {\tt r1} to
non-zero values to indicate failure. 

\begin{figure}
\caption{ \label{syscall-table} Ant-8 System Calls}
\vspace{3mm}
% $Id: ant-sys.tex,v 1.2 2002/01/02 02:13:48 ellard Exp $

\vspace{3mm}
\noindent
\begin{tabular}{|l|c|p{4.75in}|}
\hline
{\bf Service}   & {\bf Code}    & {\bf Description} \\
\hline
\hline
{\tt SysHalt}		& 0     & Halt the processor.   \\
{\tt SysDump}		& 1     & Dump core to file {\tt ant.core}.   \\
\hline
{\tt SysPutChar}	& 2     & Print the contents of {\em reg} as an
                                        ASCII character.  \\
{\tt SysGetChar}	& 3     &
                                Read a character into {\em reg}.  {\em reg}
                                must not be {\tt r0} or {\tt r1}.  If
                                EOF, {\tt r1} is set to 1. \\
\hline
{\tt SysPutInt}		& 4     & Print the contents of {\em reg} as a signed
				decimal number.
                                        \\
{\tt SysGetInt}		& 5     &
                                Read a line of input, interpret it as a
				decimal integer, and place its value
				into {\em reg}.  {\em reg}
                                must not be {\tt r0} or {\tt r1}.  If
                                EOF, {\tt r1} is set to 1.  If the line
				is too long, or empty, {\tt r1} is set to 2.
				If the line contains extraneous characters
				that are not part of the number, {\tt r1}
				is set to 3.  If the number is too large
				or too small to be represented as a signed
				8-bit number, {\tt r1} is set to 4. \\
\hline
{\tt SysPutStr}		& 6     & Print the 0-terminated {\sc ASCII}
				string that starts at {\em reg}.
				If the string is unterminated, a fault
				occurs and the ANT halts.
                                        \\
\hline
\end{tabular}


\end{figure}

%%% end of ant.tx

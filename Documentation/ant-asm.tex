% $Id: ant-asm.tex,v 1.5 2002/04/16 01:02:37 ellard Exp $

\section{The Ant-8 Assembler Reference}

\subsection{Comments}

A comment begins with a \verb$#$ and continues until the following
end-of-line.  The only exception to this is when the \verb$#$ character
appears as part of an ASCII character constant (as described in
section \ref{data-const-sec}).

\subsection{The {\tt \_data\_} Label}

A special label, {\tt \_data\_}, is used to mark the boundary between
the instructions of the program (which must appear before the {\tt
\_data\_} label) and the data of the program (which appear afterward).

\subsection{Constants}
\label{data-const-sec}

Several Ant-8 assembly instructions contain 8-bit or 4-bit constants.

The 8-bit constants can be specified in a variety of ways:
as decimal, octal, hexadecimal, or binary numbers, {\sc ASCII} codes (using
the same conventions as C), or labels.  Examples are shown in the
following table:

\begin{center}
\begin{tabular}{|l|l|l|}
\hline
Representation	& Value	& Decimal Value \\
\hline
{\em Decimal (base 10)}		&	{\tt 65}	&	65 \\
{\em Hexadecimal (base 16)}	&	{\tt 0x41}	&	65 \\
{\em Octal (base 8)}		&	{\tt 0101}	&	65 \\
{\em Binary (base 2)}		&	{\tt 0b01000001}&	65 \\
{\em {\sc ASCII}}		&	{\tt 'A'}	&	65 \\
\hline
{\em Decimal (base 10)}		&	{\tt 10}	&	10 \\
{\em Hexadecimal (base 16)}	&	{\tt 0xa}	&	10 \\
{\em Octal (base 8)}		&	{\tt 012}	&	10 \\
{\em Binary (base 2)}		&	{\tt 0b1010}	&	10 \\
{\em {\sc ASCII}}		&	{\tt '\verb$\$n'}	&	10 \\
\hline
\end{tabular}
\end{center}
\vspace{3mm}

The value of a label is the index of the subsequent instruction in
instruction memory for labels that appear in the code, or the index of
the subsequent {\tt .byte} item for labels that appear in the data.

The 4-bit constants must be specified as unsigned numbers (using
decimal, octal, hexadecimal, or binary notation).  ASCII constants or
labels cannot be used as 4-bit constants, even if their value can be
represented in 4 bits.

\subsection{Symbolic Constants}
\label{data-symconst-sec}

Constants can be given symbolic names via the {\tt .define} directive. 
This can result in substantially more readable code.  The first
operand of the {\tt .define} directive is the symbolic name for the
constant, and the second value is an integer constant.  The integer
constant must not be a label or another symbolic constant, however.

\vspace{3mm}
{\codesize
\begin{verbatim}
        .define ROWS, 10        # Defining ROWS to be 10
        .define COLS, 10        # Defining COLS to be 10

        lc      r2, ROWS        # Using ROWS as a constant
        inc     r3, COLS        # Using COLS as a constant
\end{verbatim}}
\vspace{3mm}

\subsection{The {\tt .byte} Directive}
\label{data-directive-sec}
\label{byte-figure}

The {\tt .byte} directive is used to specify data values to be
assembled into the next available locations in memory.

\vspace{3mm}
\noindent
\begin{tabular}{|ll|p{4.0in}|}
\hline
{\bf Name}      & {\bf Parameters}      & {\bf Description}     \\
\hline
{\tt .byte}     & {\em byte1, $\cdots$, byteN }   &
		Assemble the given bytes (8-bit integers) into the
		next available locations in the data segment.  As many
		as 8 bytes can be specified on the same line.  Bytes
		may be specified as hexadecimal, octal, binary, decimal
		or character constants (as described in
		\ref{data-const-sec}).

		No more than 8 bytes can be defined with the same
		{\tt .byte} statement.
                \\
\hline
\end{tabular}
\vspace{3mm}


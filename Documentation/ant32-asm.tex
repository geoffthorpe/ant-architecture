% $Id: ant32-asm.tex,v 1.4 2002/04/17 20:05:58 ellard Exp $

\section{Comments and Whitespace}

A comment begins with a \verb$#$ and continues until the following
end-of-line.  The only exception to this is when the \verb$#$
character appears as part of an ASCII character constant (as described
in section \ref{data-const-sec}).

Once comments have been removed, any line that is not indented defines
a {\em label}.  The name of the label begins with the first character
of the line, and continues until a colon ({\tt :}) has been reached. 
All other lines must be indented.  The recommended level of
indentation is at least one tab-stop; additional indentation may be
used, at the discretion of the programmer, to clarify the program
structure.

\section{Summary of Directives}

\vspace{3mm}
\noindent
\begin{tabular}{|ll|p{4.0in}|}
\hline
{\bf Name}      & {\bf Parameters}      & {\bf Description}     \\
\hline
{\tt .text}	&					&
		Assemble the following assembly language statements
		as program instructions.  (This is the default.)
		\\
{\tt .data}	&					&
		Assemble the following assembly language statements
		as data.
		\\
\hline
{\tt .define}	& {\em name}, {\em value}		&
		Bind the {\em value} to the {\em name}.
		\\
\hline
{\tt .byte}     & {\em byte1, $\cdots$, byteN }		&
		Assemble the given byte values.
		\\
{\tt .word}     & {\em word1, $\cdots$, wordN }		&
		Assemble the given word (4-byte) values.
		\\
{\tt .ascii}    & {\tt "{\em string}"}			&
		Assemble the given string.  The string is
		not zero-terminated.
		\\
{\tt .asciiz}   & {\tt "{\em string}"}			&
		Assemble the given string, including the
		a zero-terminating byte.
		\\
\hline
{\tt .align}	& {\em size}				&
		Force alignment to the next address
		used by the assembler to the given {\em size},
		skipping over memory if needed.
		\\
\hline
\end{tabular}
\vspace{3mm}

% \section{The {\tt .text} and {\tt .data} Directives}

\section{Constants}
\label{data-const-sec}

Several Ant-32 assembly instructions contain 8, 16, or 32-bit constants.
A 32-bit constant can be specified in a variety of ways:
as decimal, octal, hexadecimal, or binary numbers, {\sc ASCII} codes (using
the same conventions as C), or labels.  Examples are shown in the
following table:

\begin{center}
\begin{tabular}{|l|l|l|}
\hline
Representation	& Value	& Decimal Value \\
\hline
{\em Decimal (base 10)}		&	{\tt 65}	&	65 \\
{\em Hexadecimal (base 16)}	&	{\tt 0x41}	&	65 \\
{\em Octal (base 8)}		&	{\tt 0101}	&	65 \\
{\em Binary (base 2)}		&	{\tt 0b01000001}&	65 \\
{\em {\sc ASCII}}		&	{\tt 'A'}	&	65 \\
\hline
{\em Decimal (base 10)}		&	{\tt 10}	&	10 \\
{\em Hexadecimal (base 16)}	&	{\tt 0xa}	&	10 \\
{\em Octal (base 8)}		&	{\tt 012}	&	10 \\
{\em Binary (base 2)}		&	{\tt 0b1010}	&	10 \\
{\em {\sc ASCII}}		&	{\tt '\verb$\$n'}	&	10 \\
\hline
\end{tabular}
\end{center}
\vspace{3mm}

The value of a label is the index of the subsequent instruction in
instruction memory for labels that appear in the code, or the index of
the subsequent {\tt .byte}, {\tt .word}, or {\tt .ascii} item for
labels that appear in the data.

The 8 and 16-bit constants can be specified in all the same ways as
the 32-bit constants {\em except} for labels, which are always 32
bits.

\section{Symbolic Constants}
\label{data-symconst-sec}
\index{.define}

Constants can be given symbolic names via the {\tt .define} directive. 
This can result in substantially more readable code.  The first
operand of the {\tt .define} directive is the symbolic name for the
constant, and the second value is an integer constant.  Unfortunately,
the integer constant must not be a label or another symbolic constant.

\vspace{3mm}
{\codesize
\begin{verbatim}
        .define ROWS, 10        # Defining ROWS to be 10
        .define COLS, 10        # Defining COLS to be 10

        lc      g2, ROWS        # Using ROWS as a constant
        addi    g3, g3, COLS    # Using COLS as a constant
\end{verbatim}}
\vspace{3mm}

Note that {\tt .define}'d constants can be redefined at any point.

\section{The {\tt .byte}, {\tt .word}, and {\tt .ascii} Directives}
\label{data-directive-sec}
\label{byte-figure}

The {\tt .byte} and {\tt .word} directives are used to specify data
values to be assembled into the next available locations in memory.
{\tt .byte} is used to assemble bytes, and {\tt .word} is used to
assemble 32-bit values.

\vspace{3mm}
\noindent
\begin{tabular}{|ll|p{4.0in}|}
\hline
{\bf Name}      & {\bf Parameters}      & {\bf Description}     \\
\hline
{\tt .byte}     & {\em byte1, $\cdots$, byteN }   &
		Assemble the given bytes (8-bit values) into the
		next available locations in the data segment.  As many
		as 8 bytes can be specified on the same line.  Bytes
		may be specified as hexadecimal, octal, binary, decimal
		or character constants.
                \\
{\tt .word}     & {\em word1, $\cdots$, wordN }   &
		Assemble the given words (32-bit values) into the
		next available locations in the data segment.  As many
		as 8 words can be specified on the same line.  Words
		may be specified as labels, hexadecimal, octal, binary, decimal
		or character constants.
                \\
{\tt .ascii}	& {\tt {\em "string"}}		&
		Assemble the given string (which must be enclosed
		in double quotes) as a sequence of 8-bit ASCII values.
		Note that a terminating zero is {\em not}
		added to string by {\tt .ascii}, and must be
		placed there explicitly if desired.
		\\
{\tt .asciiz}	& {\tt {\em "string"}}		&
		Assemble the given string (which must be enclosed
		in double quotes) as a sequence of 8-bit ASCII values.
		Unlike {\tt .ascii},
		a terminating zero byte is added to the end of the string.
		\\

\hline
\end{tabular}
\vspace{3mm}

\section{{\tt .align}}
\index{.align}

The Ant-32 architecture only allows memory references that are {\em
aligned} according to their size:  4-byte word reads and writes must
always be aligned on 4-byte boundaries (their address must always be
divisible by 4).  Byte reads and writes do not have any alignment
restrictions, since all addresses are divisible by 1.

The {\tt .align} directive is used to ensure that an address is
divisible by an arbitrary amount.  The {\tt .align} directive is used
to ensure that addresses are properly aligned.  The {\tt .align}
directive causes the assembler to skip to the next address which is a
multiple of its argument {\em size}.  (If the current address is a
multiple of the {\em size}, then no skip is needed.)

For example, to ensure that the address of a {\tt .word}
is aligned in a 4-byte boundary after an {\tt .ascii} string:

\begin{verbatim}
        .ascii  "hello"
        .align  4	# make sure that xxx is aligned on a word boundary
xxx :   .word   100
\end{verbatim}

This will ensure the address {\tt xxx} is aligned on a 4-byte boundary.

{\tt .align} can also be used to align on other boundaries, such as
page boundaries (by using a size of 4096).

Note that the alignment adjustment is done {\em after} the rest of the
line is processed, and therefore it is usually incorrect to
put a label definition on the same line as a {\tt .align}, because
the label will be assigned to a possibly misaligned address.  For example:

\begin{verbatim}
xxx:    .align  4       # WRONG: xxx might not be aligned
        .word   100     # xxx might not be the address of this word.

        .align  4       # RIGHT: yyy will be aligned properly
yyy:    .word   100     # yyy will be the address of this word
\end{verbatim}


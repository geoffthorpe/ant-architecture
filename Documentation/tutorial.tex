% 02/20/94
% $Id: tutorial.tex,v 1.18 2002/04/22 16:32:22 ellard Exp $
%

\chapter{An {\sc Ant-8} Tutorial}

This section is a tutorial for {\sc Ant-8} assembly language
programming and the {\sc Ant-8} environment.
This chapter covers the basics of {\sc Ant-8} assembly language,
including arithmetic operations, simple I/O, conditionals, loops,
and accessing memory.

\section{What is Assembly Language?}
\index{assembly}

Computer instructions are represented, in a computer, as sequences of
bits.  Generally, this is the lowest possible level of representation
for a program-- each instruction is equivalent to a single,
indivisible action of the CPU.  This representation is called {\em
machine language}, and it is the only form that can be ``understood''
directly by the computer.

A slightly higher-level representation (and one that
is much easier for humans to use) is called {\em assembly language}.
Assembly language is very closely related to machine language,   
and there is usually a straightforward way to translate
programs written in assembly language into machine language.
(This translation is usually implemented by a program called
an {\em assembler}.)
Assembly language is usually a direct translation of the
machine language; one instruction in assembly language
corresponds to one instruction in the machine language.

Because of the close relationship between machine and assembly
languages, each different machine architecture usually has its own assembly
language (in fact, a particular architecture may have several),
and each is unique.

\section{Getting Started with {\sc Ant-8} Assembly: {\tt add.asm}}
\index{add.asm}

To get our feet wet, we'll write an assembly language program named
{\tt add.asm} that computes the sum of 1 and 2.  Although this task is
simple, in order to accomplish it we will need to explore several key
concepts in {\sc Ant-8} assembly language programming.

\subsection{Registers}

Like many modern CPU architectures, the {\sc Ant-8} CPU can only operate
directly on data that is stored in special locations called {\em
registers}.  The {\sc Ant-8} architecture has 16 registers, named {\tt r0}
through {\tt r15}.  Each of these registers can hold a single value.
Two of these registers have special purposes: 
register zero ({\tt r0}) always contains the value zero, and register
one ({\tt r1}) is used to hold useful values computed as part of the
most recently executed instruction.

While most modern computers have many megabytes of memory, it is
unusual for a computer to have more than a few dozen registers.  Since
most computer programs use much more data than can fit into these
registers, it is usually necessary to juggle the data back and forth
between memory and the registers, where it can be operated upon by the
CPU.  (The first few programs that we write will only use registers,
but in section \ref{load-store-sec} the use of memory is introduced.)

\subsection{Commenting}
\index{commenting}

Before we start to write the executable statements of
our program, however, we'll need to write a comment that describes
what the program is supposed to do.  In the {\sc Ant-8} assembly language,
any text between a pound sign ({\tt \#}) and the subsequent newline
is considered to be a comment, and is ignored by the assembler.
Good comments are absolutely essential!
Assembly language programs are notoriously difficult to read
unless they are well organized and properly documented.

Therefore, we start by writing the following:

{\codesize
\begin{verbatim}
# Dan Ellard -- 11/2/96
# add.asm-- A program that computes the sum of 1 and 2,
#       leaving the result in register r2.
# Registers used:
# r2 - used to hold the result.

# end of add.asm
\end{verbatim}}

Even though this program doesn't actually do anything yet, at least
anyone reading our program will know what this program is supposed to do,
and who to blame if it doesn't work\footnote{
You should put your own name on your own programs, of course;
Dan Ellard shouldn't take all the blame.}.

Unlike programs written in higher level languages, it is usually
appropriate to comment every line of an assembly language program,
often with seemingly redundant comments.  Uncommented code that seems
obvious when you write it will be baffling a few hours later.  While a
well-written but uncommented program in a high level language might be
relatively easy to read and understand, even the most well-written
assembly code is unreadable without appropriate comments.  Some
programmers prefer to add comments that paraphrase the steps performed
by the assembly instructions in a higher-level language.

We are not finished commenting this program,
but we've done all that we can do until we know a little more about
how the program will actually work.

\subsection{Finding the Right Instructions}

Next, we need to figure out what instructions the computer will need
to execute in order to add two numbers.  Since the {\sc Ant-8} architecture
has very few instructions, it won't be long before you have memorized
all of the instructions that you'll need, but as you are getting
started you'll need to spend some time browsing through the list of
instructions, looking for ones that you can use to do what you want. 
A table of all of the instructions is given in figure
\ref{mnemonic-table}, on page \pageref{mnemonic-table}.

Luckily, as we look through the list of instructions, the very first
instruction we come across is the {\tt add} instruction, which adds
two numbers together.

The {\tt add} instruction takes three operands, which must appear
in the following order:

\begin{enumerate}

\item A register that will be used to hold the result of the addition. 
	For our program, this will be {\tt r2}.

\item A register that contains the first number to be added. 
	Therefore, we're going to have to place the value 1 into a
	register before we can use it as an operand of {\tt add}. 
	Checking the list of registers used by this program (which is
	an essential part of the commenting) we select {\tt r3}, and
	make note of this in the comments.

\item A register that holds the second number to be added.  We're also
	going to have to place the value 2 into a register before we
	can use it as an operand of {\tt add}.  Checking the list of
	registers used by this program we select {\tt r4}, and make
	note of this in the comments.

\end{enumerate}

We now know how we can add the numbers, but we have to figure out how
to place 1 and 2 into the appropriate registers.  To do this, we can
use the {\tt lc} (load constant value) instruction, which places an
8-bit constant into a register.  Therefore, we arrive at the following
sequence of instructions:

{\codesize
\begin{verbatim}
# Dan Ellard -- 11/2/96
# add.asm-- A program that computes the sum of 1 and 2,
#       leaving the result in register r2.
# Registers used:
# r2 - used to hold the result.
# r3 - used to hold the constant 1.
# r4 - used to hold the constant 2.

        lc      r3, 1           # r3 = 1
        lc      r4, 2           # r4 = 2
        add     r2, r3, r4      # r2 = r3 + r4.

# end of add.asm
\end{verbatim}}

\subsection{Completing the Program}

These three instructions perform the calculation that we want, but
they do not really form a complete program.  We have told the
processor what we want it to do, but we have not told it to stop
after it has done it!

{\sc Ant-8} programs always begin executing at the first instruction in
the program.  There is no rule for where the program ends, however,
and if not told otherwise the {\sc Ant-8} processor will read past the
end of the program, interpreting whatever it finds as instructions and
trying to execute them.  It might seem sensible (or obvious) that the
processor should stop executing when it reaches the ``end'' of the
program (in this case, the {\tt add} instruction on the last line), but
there are some situations where we might want the program to continue
past the ``end'' of the program, or stop before it reaches the end.
Therefore, the {\sc Ant-8} architecture contains an instruction named
{\tt hlt} that {\em halts} the processor.

The {\tt hlt} instruction does not take any operands.
(For more information about {\tt hlt}, consult Figure
\ref{mnemonic-table} on page \pageref{mnemonic-table}.)

\index{add.asm (complete listing)}
% $Id: add.tex,v 1.6 2001/06/21 15:17:39 ellard Exp $

\renewcommand{\INSTmnemonic}	{add}
\renewcommand{\INSTdesc}	{Addition}
\renewcommand{\INSTopcode}	{0x80}
\renewcommand{\INSToptype}	{ThreeReg}
\renewcommand{\INSTfieldA}	{des}
\renewcommand{\INSTfieldB}	{src1}
\renewcommand{\INSTfieldC}	{src2}

\renewcommand{\INSTsemantic}	{
	\Reg{des} $\leftarrow$ \Reg{src1} + \Reg{src2}
}

\renewcommand{\INSTexceptions}	{}



\subsection{The Format of {\sc Ant-8} Assembly Programs}

As you read {\tt add.asm}, you may notice several formatting
conventions-- every instruction is indented,
and each line contains at most one instruction.  These conventions are
{\em not} simply a matter of style, but are actually part of the
definition of the {\sc Ant-8} assembly language.

The first rule of {\sc Ant-8} assembly formatting is that instructions {\em
must} be indented.  Comments do not need to be indented, but all of
the code itself must be.  The second rule of {\sc Ant-8} assembly formatting
is that only one instruction can appear on each line.  (There are a
few additional rules, but these will not be important until section
\ref{Labels-subsec}.)

Unlike many programming languages, where the use of whitespace and
formatting is largely a matter of style, in {\sc Ant-8} assembly language some
use of whitespace is required.

\subsection{Running {\sc Ant-8} Assembly Language Programs}

At this point, we should have a working program.  Now, it's time to
run it and see what happens.

There are two principal ways of running an {\sc Ant-8} program-- using
the commandline tools ({\tt aa8}, {\tt ad8} and {\tt ant8}), or using
{\sc AIDE8} (the {\sc Ant-8} Integrated Development Environment).

\subsubsection{Using the Commandline Tools}

Before the commandline tools can run on a program, the program
must be written in a file.  This file must be plain text, and by
convention {\sc Ant-8} assembly language files have a suffix of {\tt .asm}.
In this example, we will assume that the file {\tt add.asm} contains
a copy of the {\tt add} program listed earlier.

Before we can run the program, we must {\em assemble} it.  The
assembler translates the program from the assembly language
representation to the machine language representation.  The assembler
for {\sc Ant-8} is called {\tt aa8}, so the appropriate command would be:

\begin{verbatim}
        aa8 add.asm
\end{verbatim}

This will create a file named {\tt add.ant} that contains
the {\sc Ant-8} machine-language representation of the program in
{\tt add.asm}.

Now that we have the assembled version of the program,
we can test it by loading it into the {\sc Ant-8} debugger in
order to execute it.
The name of the {\sc Ant-8} debugger is {\tt ad8}, so to run the debugger,
use the {\tt ad8} command followed
by the name of the machine language file to load.
For example, to run the program that we just wrote
and assembled:

\begin{verbatim}
        ad8 add.ant
\end{verbatim}

After starting, the debugger will display the following
prompt: {\tt >>}.
Whenever you see the {\tt >>} prompt, you know that the debugger
is waiting for you to specify a command for it to execute.

Once the program is loaded, you can use the {\tt r} (for {\em run})
command to run it:
\begin{verbatim}
        >> r
\end{verbatim}

The program runs, and then the debugger indicates that it is ready to
execute another command.  Since our program is supposed to
leave its result in register {\tt r2}, we can verify that the
program is working by asking the debugger to print out the contents of
the registers using the {\tt p} (for {\em print}) command,
to see if it contains the result we expect:

{\codesize
\begin{verbatim}
>> p 
 r01  r02  r03  r04  r05  r06  r07  r08  r09  r10  r11  r12  r13  r14  r15
  00   03   01   02   00   00   00   00   00   00   00   00   00   00   00
   0    3    1    2    0    0    0    0    0    0    0    0    0    0    0
\end{verbatim}}

The {\tt p} command displays the contents of each register.
The first line lists the register names.  The following line
lists the value of each register in hexadecimal, and the
last line lists the same number in decimal.

The {\tt q} command exits the debugger.

{\tt ad8} includes a number of features that will make debugging your
{\sc Ant-8} assembly language programs much easier.  Type {\tt h} (for {\em help})
at the
{\tt >>} prompt for a full list of the {\tt ad8} commands, or consult the
{\tt ad8} documentation for more information.

\subsubsection{Using {\sc AIDE8}}

{\sc AIDE8} provides a more tightly integrated way of creating,
debugging, and running {\sc Ant-8} programs.  Although it is less
flexible than running each of the tools individually, for most users
it is more than sufficient.

When {\sc AIDE8} starts, only the editor window is shown.  This window
can be used to create or edit an {\sc Ant-8} program.  After the program
is written, it can be assembled by pressing the {\bf Assemble} button. 
If the assembly process is successful, no error messages will be
displayed; otherwise, the cause of the error is printed at the bottom
of the screen and the offending line of the program is highlighted.

Once the program has been written and assembled, pressing the {\bf
Debug} button brings up the debugger window.  The debugger window
displays the complete state of the {\sc Ant-8} machine.  (The debug
window can be displayed at any time, even if there isn't a program
loaded into the {\sc Ant-8} machine, but without a program to display
there isn't much to see.)

To simply run the program, click on the {\bf Run} button in the
upper-left corner of the debugger window.  The execution of the
program, instruction by instruction, will be displayed in the debugger
window as the state of the processor is updated.

The {\sc AIDE8} contains many other features.  Consult the {\bf Help}
menu of {\sc AIDE8} for more information.


\section{Reading and Printing: {\tt add2.asm}}
\index{add2.asm}
\label{add2-sec}

Our program to compute $1 + 2$ is not particularly useful, although
it does demonstrate a number of important details about programming
in {\sc Ant-8} assembly language and the {\sc Ant-8} environment.  For our next
example, we'll write a program named {\tt add2.asm} that computes
the sum of two numbers specified by the user at runtime,
and displays the result on the screen.

The algorithm this program will use is:
\begin{enumerate}

\item   Read the two numbers from the user.

        We'll need two registers to hold these two numbers.  We will use
        {\tt r3} and {\tt r4} for this.

\item   Compute their sum.

        We'll need a register to hold the result of this addition.  We
        can use {\tt r2} for this.

\item   Print the sum, followed by a newline.

\item   Halt.

	We already know how to do this, using {\tt hlt}.

\end{enumerate}

The only parts of the algorithm that we don't know how to do yet are
to read the numbers from the user, and print out the sum. 

{\sc Ant-8} does its I/O (or ``input/output'') using the {\tt in}
instruction to read values from the user into the computer, and
the {\tt out} instruction to display values to the user.

The {\tt in} instruction allows the user to specify values in
one of three different formats:  hexadecimal, binary, or
{\sc ASCII}.  Similarly, the {\tt out} instruction can display a
value in hexadecimal, binary, or {\sc ASCII}.  Note that there is no
way to directly input or output a decimal value!

This gives the following program:

\index{add2.asm (complete listing)}
\input{Tutorial/add2}

\section{Branches, Jumps, and Conditional Execution: {\tt larger.asm}}
\index{branching}
\index{larger.asm}

The next program that we will write will read two numbers from the
user, and print out the larger of the two.  The basic structure of
this program is similar to the one used by {\tt add2.asm}, except
that we're computing the maximum rather than the sum of two numbers. 
The difference is that the behavior of this program depends upon the
input, which is unknown when the program is written.
The program must be able to
decide whether to execute instructions to print out the first number
or execute the instructions to print out the second number at runtime. 
This is known as {\em conditional execution}-- whether or not certain
parts of program are executed depends on a condition that is not known
when the program is written.

Browsing through the instruction set again, we find a description of
the {\sc Ant-8} branching instructions.  These allow the programmer to specify
that execution should {\em branch} (or {\em jump}) to a location other
than the next instruction.  These instructions allow conditional
execution to be implemented in assembly language (although in not
nearly as clean a manner as higher-level languages provide).

In {\sc Ant-8} assembler, there are three branching instructions:  {\tt
bgt}, {\tt beq} and {\tt jmp}.  {\tt bgt} and {\tt beq} are called
{\em conditional} branches, because they cause the program to branch
when a specific condition holds.  In contrast, the {\tt jmp}
instruction is an {\em unconditional} branch, which is always taken.

The {\tt bgt} instruction takes three registers as arguments.  If the
number in the second register is larger than the number in the third,
then execution will jump to the location specified by the first;
otherwise it continues at the next instruction.

The {\tt beq} instruction is similar in form to the {\tt bgt}
instruction, except that the branch occurs if the second and third
registers contain the same value.

The {\tt jmp} instruction takes a single argument, which is an
unsigned 8-bit constant.  Execution jumps to the location specified by
the constant.

\subsection{Labels}
\label{Labels-subsec}
\index{labels}

Keeping track of the numeric addresses in memory of the instructions
to which we want to branch is troublesome and tedious at best-- a small
error can make our program misbehave in strange ways, and if we change
the program at all by inserting or removing instructions, we will have
have to carefully recompute all of these addresses and then change all
of the instructions that use these addresses.  This is much more than
most humans can possibly keep track of.  Luckily, the computer is very
good at keeping track of details like this, so the {\sc Ant-8} assembler
provides {\em labels}, a human-readable shorthand for addresses.

A {\em label} is a symbolic name for an address in memory.  In {\sc Ant-8}
assembler, a {\em label definition} is an identifier followed by a colon.
{\sc Ant-8} identifiers use the same conventions as Python, Java, C, C++, and
many other contemporary languages:

\begin{itemize}

\item {\sc Ant-8} identifiers must begin with an underscore, an uppercase
	character (A-Z) or a lowercase character (a-z).

\item Following the first character there may be zero or more
	underscores, or uppercase, lowercase, or numeric (0-9)
	characters.  No other characters can appear in an identifier.

\item Although there is no intrinsic limit on the length of {\sc Ant-8}
	identifiers, some {\sc Ant-8} tools may reject identifiers longer than
	100 characters.

\end{itemize}


Labels must be the first item on a line, and must begin in the ``zero
column'' (immediately after the left margin).  Label definitions {\em
cannot} be indented, but all other non-comment lines {\em must} be.

Since labels must begin in column zero, only one label definition is
permitted on each line of assembly language, but a location in memory
may have more than one label.  Giving the same location in memory more
than one label can be very useful.  For example, the same location in
your program may represent the end of several nested ``if''
statements, so you may find it useful to give this instruction several
labels corresponding to each of the nested ``if'' statements.

When a label appears alone on a line, it refers to the following
memory location.  This is often good style, since it allows the use of
long, descriptive labels without disrupting the indentation of the
program.  It also leaves plenty of space on the line for the
programmer to write a comment describing what the label is used for,
which is very important since even relatively short assembly language
programs may have a large number of labels.

\subsection{Branching Using Labels}
\index{branch with labels }

Using the branching instructions and labels we can do what we want in
the {\tt larger.asm} program.  Since the branching instructions take a
register containing an address as their first argument, we need to
somehow load the address represented by the label into a register.  We
do this by using the {\tt lc} ({\em load constant}) command.  The {\tt
larger.asm} program illustrates how this is done.

\input{Tutorial/larger}

Note that since {\sc Ant-8} does not have an instruction to {\em copy} or {\em
move} the contents of one register to another, in order to copy the
value of one register to another register we've added 0 to one
register and put the sum in the destination register in order to
achieve the desired result.  (Recall that register {\tt r0} always
contains the constant zero.)

\section{Looping: {\tt loop.asm}}
\index{looping}
\index{loop.asm}
\label{loop-sec}

In the previous example program, we used the jump and branch
instructions to implement conditional execution, which meant that we
could skip over some instructions depending on the values that the
user typed.  We can also use these instructions to implement {\em
loops}, which allow the program to repeatedly execute a sequence of
instructions an arbitrary number of times.

The next program that we will write will read a character $A$ (as
{\sc ASCII}) and then a number $B$ (as hexadecimal) from the user, and then
print $B$ copies of the character $A$.  This algorithm translates
easily into {\sc Ant-8} assembler; the only thing that is new is that the
execution might jump ``backwards'' in the program to repeat some
instructions more than once.  The {\tt loop.asm} programs shows how
this is done.

\input{Tutorial/loop}


\section{Strings: {\tt hello.asm}}
\index{hello.asm}
\label{hello-sec}
\label{load-store-sec}

The next program that we will write is the ``Hello World'' program,
a program that simply prints the message ``Hello World'' to the screen
and then halts.

There is no way in {\sc Ant-8} to print out more than one character at a
time, so we must use a loop to print out each character of the string,
starting at the beginning and continuing until we reach the end of the
string.


The string ``{\tt Hello World}'' is not part of the instructions of
the program, but it is part of the memory used by the program.  The
assembler places all data values (not instructions) after all of the
instructions in memory.

The value of the data memory is loaded into memory at assembly time. 
You will have to be careful to not accidently overwrite your data
during run-time!

The way that the initial contents of data memory are defined is via
the {\tt .byte} directive.  {\tt .byte} looks like an instruction
which takes as many as eight 8-bit constants, but it is not an
instruction at all.  Instead, it is a directive to the assembler to
fill in the next available locations in memory with the given
values.

All of the {\tt .byte} items in an {\sc Ant-8} program must appear at
the end of the program, after the special label {\tt \_data\_}.  The
{\tt \_data\_} label indicates to the assembler that all subsequent
items are data.  No instructions are permitted after the {\tt \_data\_}
label.

In our programs, we will use the following convention for {\sc ASCII}
strings:  a {\em string} is a sequence of characters terminated by a 0
byte.  For example, the string ``hi'' would be represented by the
three characters `h', `i', and 0.  Using a 0 byte to mark the end
of the string is a convenient method, used by several contemporary
languages.

The program {\tt hello.asm} is an example of how to use labels and
treat characters in memory as strings:

\index{hello.asm (complete listing)}
\input{Tutorial/hello}

The label {\tt str\_data} is the symbolic representation of the
memory location where the string begins in data memory.

\section{Character I/O: {\tt echo.asm}}
\index{echo.asm}
\label{echo-sec}
\index{character I/O}

Now that we have mastered loops and reading and printing integers,
we'll turn our attention to reading and printing single characters. 
(We've already seen how to read and write numbers, and the process is
similar, except that we use the {\sc ASCII} device instead of {\tt
Hex}.)

The program we'll write in this section simply echos whatever you type
to it, until EOI ({\em end of input}) is reached.

The way that EOI is detected in {\sc Ant-8} is that when the EOI is
reached, any attempt to use {\tt in} to read more input will
immediately fail, and a non-zero value will be placed in register {\tt
r1} to indicate that there was an error.  If the {\tt in} succeeds,
then {\tt r1} is set to zero. 

Therefore, our program will loop, continually using {\tt in} to read
characters, and checking {\tt r1} after each {\tt in} to see whether
or not the EOI has been reached.

\input{Tutorial/echo}

{\bf Note:} because of the difference between the user interface of
the debugger in {\sc AIDE8} and the ordinary runtime {\sc Ant-8}
environment, {\tt echo.asm} behaves differently when run under {\sc
AIDE8} than when run via {\tt ant8} or {\tt ad8}.  This is because in
{\sc AIDE8}, every input, including {\sc ASCII} input, must be followed
by a newline, while in ordinary operation {\sc ASCII} input does not. 
This should be considered a shortcoming of {\sc AIDE8}, not a problem
with {\tt echo.asm}.

\section{Bit Operations: {\tt shout.asm}}

The next program we shall write is very similar to {\tt echo.asm},
except that instead of simply echoing its input, it converts all of
the lowercase characters in the input to uppercase in the output.

There are several ways that this could be accomplished.  The easiest
way, and one that does not require learning about any new {\sc Ant-8}
instructions, would be to note that all of the lowercase characters in
{\sc ASCII} are arranged consecutively, starting with {\tt 'a'} (with
a value of {\tt 0x61}) and continuing through {\tt 'z'} (with a value
of {\tt 0x7A}).  The uppercase characters are arranged in the same
manner, ranging from {\tt 'A'} = {\tt 0x41} to {\tt 'Z'} = {\tt 0x5A}. 
Therefore, in order to convert from uppercase to lowercase all we need
to do is take any characters that have {\sc ASCII} codes in the range
{\tt 0x61} to {\tt 0x7A} and subtract {\tt 0x20} from them before
printing them.

However, our goal in this section is to learn about {\sc Ant-8}'s
bitwise instructions, and so we use a different observation-- the {\tt
ASCII} code for the lowercase characters and the corresponding
uppercase characters differ only in a single bit.  All the lowercase
characters have the bit corresponding to {\tt 0x20} set to 1, while
all the uppercase characters do not.  Therefore, if we have a
lowercase character, in order to convert it to uppercase all we need to
do is set this bit to 0.  The bit we are interested in is the fifth
bit (counting from the right and starting at zero).

To change the fifth bit from 1 to 0, we can use the {\tt and}
instruction.  If the 8-bit value is in register {\tt r2}, then
computing the bitwise {\sc And} of the value of {\tt r2} and the 8-bit
value value consisting of all 1 bits {\em except} the fifth bit, the
result will be identical to the original value of {\tt r2} except that
the fifth bit will be 0.

We could explicitly compute the value of the 8-bit value that has all
1-bits except for the fifth bit, or we can let the computer do this
work for us.  We choose the latter approach, because this also means
that we can introduce two more instructions, {\tt shf} and {\tt nor}.

Our first task is to initialize {\tt r8} with the 8-bit value
consisting of all zero bits, except for the fifth bit.  (We know this
value is {\tt 0x20}, and so we could simply use this value, but we'll
use {\tt shf} instead:

\begin{verbatim}
        lc      r8, 1           # r8 has all 0 bits, except
                                # bit zero which is 1.
        lc      r9, 5
        shf     r8, r8, r9      # shift the 1 bit over 5 spaces.
                                # r8 is now 0x20.

        nor     r8, r8, r8      # NOR r8 with itself, which changes  
                                # all the 1 bits to 0 and vice versa.
                                # r8 is now 0xDF.
\end{verbatim}

The complete program is implemented in {\tt shout.asm}.

\input{Tutorial/shout}


%%% end of tutorial.tex


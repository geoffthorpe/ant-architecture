% $Id: datarep-sec.tex,v 1.5 2002/04/17 20:05:58 ellard Exp $

\section{Introduction}

In order to understand how a computer is able
to manipulate data and perform computations,
you must first understand how data is represented by a computer.

At the lowest level, the indivisible unit of data in a computer is a
{\em bit}\index{bit}.
A bit represents a single binary value, which may be either 1 or 0.
In different contexts, a bit value of 1 and 0 may also be referred to as
``true'' and ``false'',
``yes'' and ``no'',
``high'' and ``low'',
``set'' and ``not set'', or
``on'' and ``off''.

The decision to use binary values, rather than something larger (such
as decimal values) was not purely arbitrary-- it is due
in a large part to the relative simplicity of building electronic
devices that can manipulate binary values.


\section{Representing Integers}

\subsection{Unsigned Binary Numbers}
\index{unsigned binary numbers}

%%% This really needs some explanation here.

While the idea of a number system with only two values may
seem odd, it is actually very similar to the decimal system
we are all familiar
with, except that each digit is a bit containing a 0 or 1
rather than a number from 0 to 9.
(The word ``bit'' itself is a contraction of the words ``binary digit'')
For example, figure~\ref{datarep-binary-decimal-table} shows several
binary numbers, and the equivalent decimal numbers.

\begin{figure}[hbtp]
\caption{Binary and Decimal Numbers}
\label{datarep-binary-decimal-table}
\begin{center}
\begin{tabular}{|rcr|}
\hline
{\bf Binary}    &       & {\bf Decimal} \\
\hline
       0        & =             & 0     \\
       1        & =             & 1     \\
      10        & =             & 2     \\
      11        & =             & 3     \\
     100        & =             & 4     \\
     101        & =             & 5     \\
     110        & =             & 6     \\
$\vdots$        & $\vdots$      & $\vdots$      \\
11111111        & =             & 255   \\
\hline
\end{tabular}
\end{center}
\end{figure}

In general, the binary representation of $2^{k}$ has a $1$ in
binary digit $k$ (counting from the {\em right}, starting at 0)
and a $0$ in every other digit.
(For notational convenience,
the $i$th bit of a binary number $A$ will be
denoted as $A_{i}$.)

The binary representation of a number that is not a power
of 2 has the bits set corresponding to the powers of two
that sum to the number:  for example, the decimal number
$6$ can be expressed in terms of powers of 2 as
$1 \times 2^{2} ~+~ 1 \times 2^{1} ~+~ 0 \times 2^{0}$,
so it is written in binary as $110$.

An eight-digit binary number is commonly called a {\em byte}.
In this text, binary numbers will usually be written as bytes
(i.e. as strings of eight binary digits).  For example, the binary number
$101$ would usually be written as $00000101$-- a $101$
padded on the left with five zeros, for a total of eight digits.

Whenever there is any possibility of ambiguity between
decimal and binary notation, the {\em base} of the number
system (which is 2 for binary, and 10 for decimal) is
appended to the number as a subscript.  Therefore, $101_{2}$
will always be interpreted as the binary representation for five,
and never the decimal representation of one hundred and one
(which would be written as $101_{10}$).

\subsubsection{Conversion of Binary to Decimal}
\label{datarep-binary-to-decimal-sec}

To convert an unsigned binary number to a decimal number, add
up the decimal values of the powers of 2 corresponding to bits
which are set to 1 in the binary number.
Algorithm~\ref{datarep-binary-to-decimal-alg} shows a method to do this.
Some examples of conversions from binary to decimal
are given in figure~\ref{datarep-bin-to-dec-figure}.

\begin{algorithm}{Binary to Decimal}{datarep-binary-to-decimal-alg}{
        To convert a binary number to decimal.
}

\begin{itemize}
\item   Let $X$ be a binary number, $n$ digits in length, composed of bits
                $X_{n-1} \cdots X_{0}$.
\item   Let $D$ be a decimal number.
\item   Let $i$ be a counter.
\end{itemize}

\begin{enumerate}
\item   Let $D \leftarrow 0$.
\item   Let $i \leftarrow 0$.
\item   While $i < n$ do:
        \begin{itemize}
        \item   If $X_{i}$ is $1$
                (i.e. if bit $i$ in $X$ is $1$),
                then set $D \leftarrow (D + 2^{i})$.
        \item   Set $i \leftarrow (i + 1)$.
        \end{itemize}
\end{enumerate}
\end{algorithm}

\begin{figure}[hbtp]
\caption{Examples of Conversion from Binary to Decimal}
\label{datarep-bin-to-dec-figure}
\begin{center}
\begin{tabular}{|lclclcr|}
\hline
{\bf Binary}    & & & & & & {\bf Decimal} \\
\hline

00000000        & = &   $0$                     & = &
                        $0$                     & = & 0 \\
& & & & & & \\
00000101        & = & $2^2 + 2^0$               & = &
                        $4 + 1$                 & = & 5 \\
& & & & & & \\
00000110        & = & $2^2 + 2^1$               & = &
                        $4 + 2$                 & = & 6 \\
& & & & & & \\
00101101        & = & $2^5 + 2^3 + 2^2 + 2^0$   & = &
                        $32 + 8 + 4 + 1$        & = & 45 \\
& & & & & & \\
10110000        & = & $2^7 + 2^5 + 2^4$         & = &
                        $128 + 32 + 16$         & = & 176 \\
\hline
\end{tabular}
\end{center}
\end{figure}


Since there are $2^{n}$ unique sequences of $n$ bits,
if all the possible bit sequences of length $n$ are used,
starting from zero, the largest number will be $2^{n} - 1$.

\subsubsection{Conversion of Decimal to Binary}
\label{datarep-decimal-to-binary-algorithm}

An algorithm for converting a decimal number to binary notation is
given in algorithm~\ref{datarep-decimal-to-binary-alg}.

\begin{algorithm}{Decimal to Binary}{datarep-decimal-to-binary-alg}{
        To convert a positive decimal number to binary.
}

\begin{itemize}
\item   Let $X$ be an unsigned binary number, $n$ digits in length.
\item   Let $D$ be a positive decimal number, no larger than $2^{n} - 1$.
\item   Let $i$ be a counter.
\end{itemize}
\begin{enumerate}
\item   Let $X \leftarrow 0$ (set all bits in $X$ to $0$).
\item   Let $i \leftarrow (n - 1)$.
\item   While $i \geq 0$ do:
        \begin{enumerate}
        \item   If $D \geq 2^{i}$, then
                \begin{itemize}
                \item   Set $X_{i} \leftarrow 1$
                        (i.e. set bit $i$ of $X$ to $1$).
                \item   Set $D \leftarrow (D - 2^{i})$.
                \end{itemize}
        \item   Set $i \leftarrow (i - 1)$.
        \end{enumerate}
\end{enumerate}
\end{algorithm}

\subsubsection{Addition of Unsigned Binary Numbers}
\label{datarep-unsigned-addition-sec}

Addition of binary numbers can be done in exactly the same
way as addition of decimal numbers, except that all of the
operations are done in binary (base 2) rather than
decimal (base 10).
Algorithm~\ref{datarep-unsigned-addition-alg} gives a method
which can be used to perform binary addition.

\begin{algorithm}{Unsigned Binary Addition}{datarep-unsigned-addition-alg}{
        Addition of unsigned binary numbers.
}

\begin{itemize}
\item   Let $A$ and $B$ be a pair of $n$-bit binary numbers.
\item   Let $X$ be a binary number which will hold the sum of $A$ and $B$.
\item   Let $c$ and $\hat{c}$ be carry bits.
\item   Let $i$ be a counter.
\item   Let $s$ be an integer.
\end{itemize}
\begin{enumerate}
\item   Let $c \leftarrow 0$.
\item   Let $i \leftarrow 0$.
\item   While $i < n$ do:
        \begin{enumerate}
	\item	Set $s \leftarrow A_{i} + B_{i} + c$.
        \item   Set $X_{i}$ and $\hat{c}$ according to the 
		following rules:
		\begin{itemize}
		\item	If $s$ is $0$, then
			$X_{i} \leftarrow 0$ and $\hat{c} \leftarrow 0$.
		\item	If $s$ is $1$, then
			$X_{i} \leftarrow 1$ and $\hat{c} \leftarrow 0$.
		\item	If $s$ is $2$, then
			$X_{i} \leftarrow 0$ and $\hat{c} \leftarrow 1$.
		\item	If $s$ is $3$, then
			$X_{i} \leftarrow 1$ and $\hat{c} \leftarrow 1$.
		\end{itemize}
        \item   Set $c \leftarrow \hat{c}$.
        \item   Set $i \leftarrow (i + 1)$.
        \end{enumerate}
\end{enumerate}

\end{algorithm}

When algorithm~\ref{datarep-unsigned-addition-alg} terminates,
if $c$ is not 0, then an {\em overflow}\index{overflow}
has occurred-- the resulting number is simply too large to
be represented by
an $n$-bit unsigned binary number.

\subsection{Signed Binary Numbers}
\index{signed binary numbers}

The major flaw with the representation that we've used for
unsigned binary numbers is that it doesn't include a way to
represent negative numbers.

There are a number of ways to extend the unsigned representation
to include negative numbers.
One of the easiest is to add an additional bit
to each number that is used to represent the {\em sign} of the
number-- if this bit is $1$, then the number is negative; otherwise
the number is positive (or vice versa).
This is analogous to the way that we write negative numbers
in decimal-- if the first symbol of the number is a negative sign,
then the number is negative, otherwise the number is positive.

Unfortunately, when we try to adapt the algorithm for addition
to work properly with this representation, this apparently simple
method turns out to cause some trouble.
Instead of simply adding the numbers together
as we do with unsigned numbers, we now need to consider
whether the numbers being added are positive or negative.
If one number is positive and the other negative, then we
actually need to do subtraction instead of addition, so
we'll need to find an algorithm for subtraction.  Furthermore,
once we've done the subtraction, we need to compare the
the unsigned magnitudes of the numbers to determine whether the
result is positive or negative!

Luckily, there is a representation that allows us to represent
negative numbers in such a way that addition (or subtraction)
can be done easily, using algorithms very similar to the ones
that we already have.
The representation that we will use
is called {\em two's complement} notation.
\index{two's complement}
\index{2's complement}

To introduce two's complement, we'll start by defining, in
algorithm~\ref{datarep-twos-complement-negate-alg}, the
algorithm that is used to compute the negation of a two's complement number.

\begin{algorithm}{Two's Complement Negation}
	{datarep-twos-complement-negate-alg}{
        Negation of a two's complement number.
}

\begin{enumerate}
\item   Let $\bar{x}$ be the
        {\em logical complement}\index{logical complement} of $x$.

        The logical complement (also called the {\em one's complement})
        \index{one's complement} \index{1's complement}
        is formed by flipping all the bits in the number, changing
        all of the $1$ bits to $0$, and vice versa.

\item   Let $X \leftarrow \bar{x} + 1$.

        If this addition {\em overflows}, then the overflow bit
        is discarded.
\end{enumerate}

By the definition of two's complement, the resulting $X$ is the
negation of the original $x$.

\end{algorithm}

Figure~\ref{datarep-twos-complement-neg-table}
shows the process of negating several
numbers.  Note that the negation of zero is zero.

\begin{figure}[hbtp]
\caption{Examples of Negation Using Two's Complement}
\begin{center}
\label{datarep-twos-complement-neg-table}

\begin{tabular}{llcr}
                & 00000110      & = & 6 \\
1's complement  & 11111001      &   &   \\
Add 1           & 11111010      & = & -6 \\
\\
\end{tabular}

\begin{tabular}{llcr}
                & 11111010      & = & -6 \\
1's complement  & 00000101      &   &   \\
Add 1           & 00000110      & = & 6 \\
\\
\end{tabular}

\begin{tabular}{llcr}
                & 00000000      & = & 0 \\
1's complement  & 11111111      &   &   \\
Add 1           & 00000000      & = & 0 \\
\\
\end{tabular}
\end{center}
\end{figure}

This representation has several important properties:
\begin{itemize}
\item   The leftmost (most significant)
        bit also serves as a sign bit; if 1, then the number is negative,
        if 0, then the number is positive or zero.

\item   The rightmost (least significant)
        bit of a number always determines
        whether or not the number is odd or even--
        if bit 0 is 0, then the number is even, otherwise the number is odd.

\item   The largest positive number that can be represented in
        two's complement notation in an $n$-bit binary number is $2^{n-1} - 1$.
        For example, if $n$ is $8$, then the largest positive number is
        $01111111 ~~=~~ 2^{7} - 1 ~~=~~ 127$.

\item   Similarly, the ``most negative'' number is $-2^{n-1}$, so
        if $n = 8$, then it is $10000000$, which is $-2^{7} ~~=~~ -128$.
        Note that the negative of the most negative number (in this
        case, 128) cannot be represented in this notation.
\end{itemize}


\subsubsection{Addition and Subtraction of Signed Binary Numbers}

The same addition algorithm that was used for unsigned binary numbers
also works properly for two's complement numbers.

\begin{tabular}{llcr}
        \\
        &       00000101        & = & 5         \\
  $+$   &       11110101        & = & -11       \\
\hline
        &       11111010        & = & -6        \\
        \\
\end{tabular}

Subtraction is also done in a similar way: to subtract A from B, take
the two's complement of A and then add this number to B.

The conditions for detecting overflow are different for signed
and unsigned numbers, however.  If we use
algorithm~\ref{datarep-unsigned-addition-alg} to add two unsigned numbers,
then if $c$ is $1$ when the addition terminates,
this indicates that the result has an absolute value
too large to fit the number of bits allowed.
With signed numbers, however, $c$ is not relevant, and an overflow
occurs when the signs of both numbers being added are the same
but the sign of the result is opposite.  If the two numbers being added have
opposite signs, however, then an overflow {\em cannot} occur.

For example, consider the sum of $1$ and $-1$:

\begin{tabular}{llcrl}
        \\
        &       00000001        & = & 1         & \\
  $+$   &       11111111        & = & -1        & \\
\hline
        &       00000000        & = & 0         & {\bf Correct!}        \\
        \\
\end{tabular}

In this case, the addition will overflow, but it is not
an error, since the result that we get (without considering
the overflow) is exactly correct.

On the other hand, if we compute the sum of 127 and 1, then
a serious error occurs:

\begin{tabular}{llcrl}
        \\
        &       01111111        & = & 127       &       \\
  $+$   &       00000001        & = & 1         &       \\
\hline
        &       10000000        & = & -128      & {\bf Uh-oh!}  \\
        \\
\end{tabular}

Therefore, we must be very careful when doing signed binary arithmetic
that we take steps to detect bogus results.
In general:
\begin{itemize}
\item   If $A$ and $B$ are of the same sign, but $A + B$ is of
        the opposite sign, then an overflow or wraparound error
        has occurred.
\item   If $A$ and $B$ are of different signs, then $A + B$ will
        never overflow or wraparound.
\end{itemize}

\subsubsection{Shifting Signed Binary Numbers}

Another useful property of the two's complement notation
is the ease with which numbers can be multiplied or divided by
two.  To multiply a number by two, simply shift the number ``up''
(to the left) by one bit, placing a 0 in the least significant bit.
To divide a number in half, simply shift the the number ``down''
(to the right) by one bit (but do not change the sign bit).

Note that in the case of odd numbers, the effect of shifting
to the right one bit is like dividing in half, rounded towards
negative infinity, so that 51 shifted to the right one bit becomes 25,
while -51 shifted to the right one bit becomes -26.

\begin{figure}
\caption{ Doubling and Halving Two's Complement Numbers }
\begin{center}
\begin{tabular}{llcr}
\\
                & 00000001      & = & 1 \\
Double          & 00000010      & = & 2 \\
Halve		& 00000000      & = & 0 \\
\\
                & 00110011      & = & 51 \\
Double          & 01100110      & = & 102 \\
Halve           & 00011001      & = & 25 \\
\\
                & 11001101      & = & -51 \\
Double          & 10011010      & = & -102 \\
Halve           & 11100110      & = & -26 \\
\\
\end{tabular}
\end{center}
\end{figure}

\subsubsection{Hexadecimal Notation}
\index{hexadecimal}
\index{octal}

Writing numbers in binary notation can soon get tedious,
since even relatively small numbers require many
binary digits to express.  A more compact notation, called {\em hexadecimal}
(base 16), is usually used to express large binary numbers.
In hexadecimal, each digit represents four unsigned binary digits.

Another notation, which is not as common currently, is called {\em octal}
and uses base eight to represent groups of three bits.
Figure~\ref{datarep-hex-oct-table} show examples of binary, decimal, octal, and
hexadecimal numbers.

\begin{figure}
\caption{Hexadecimal and Octal}
\label{datarep-hex-oct-table}
\begin{center}
\begin{tabular}{|l|c|c|c|c|c|c|c|c|}
\hline
Binary  & 0000  & 0001  & 0010  & 0011  & 0100  & 0101  & 0110  & 0111  \\
\hline
Decimal & 0     & 1     & 2     & 3     & 4     & 5     & 6     & 7     \\
\hline
Hex     & 0     & 1     & 2     & 3     & 4     & 5     & 6     & 7     \\
\hline
Octal   & 0     & 1     & 2     & 3     & 4     & 5     & 6     & 7     \\
\hline
\end{tabular}
\end{center}

\begin{center}
\begin{tabular}{|l|c|c|c|c|c|c|c|c|}
\hline
Binary  & 1000  & 1001  & 1010  & 1011  & 1100  & 1101  & 1110  & 1111  \\
\hline
Decimal & 8     & 9     & 10    & 11    & 12    & 13    & 14    & 15    \\
\hline
Hex     & 8     & 9     & A     & B     & C     & D     & E     & F     \\
\hline
Octal   & 10    & 11    & 12    & 13    & 14    & 15    & 16    & 17    \\
\hline
\end{tabular}
\end{center}
\end{figure}

For example, the number ${\tt 200}_{10}$ can be written as
${\tt 11001000}_{2}$,
${\tt C8}_{16}$, or
${\tt 310}_{8}$.

\section{Representing Characters}

Just as sequences of bits can be used to represent numbers,
they can also be used to represent the letters of the alphabet,
as well as other characters.

Since all sequences of bits represent numbers, one way to
think about representing characters by sequences of bits
is to choose a number that corresponds to each character.
The most popular correspondence currently is the ASCII character set.
\index{ASCII}
ASCII, which stands for the American Standard Code for Information
Interchange, uses 7-bit integers to represent characters, using the
correspondence shown in table~\ref{ascii}.

%%% Jam the ASCII table in here.
\begin{figure}[hbt]
\caption{The ASCII Character Set}
\label{ascii}
\index{ASCII}
\begin{center}
\input{ascii}
\end{center}
\end{figure}

When the ASCII character set was chosen, some care was taken to
organize the way that characters are represented in order to make
them easy for a computer to manipulate.  For example, all of the
letters of the alphabet are arranged in order, so that sorting
characters into alphabetical order is the same as sorting in
numerical order.
In addition, different classes of characters
are arranged to have useful relations.  For example, to convert
the code for a lowercase letter to the code for the same letter in
uppercase, simply set the 6th bit of the code to 0 (or subtract 32).
ASCII is by no means the only character set to have
similar useful properties, but it has emerged as the standard.

The ASCII character set does have some important limitations, however.
One problem is that the character set only defines the representations
of the characters used in written English.  This causes problems with
using ASCII to represent other written languages.  In particular,
there simply aren't enough bits to represent all the written characters
of languages with a larger number of characters (such as Chinese or
Japanese).  Already new character sets which address these problems
(and can be used to represent characters of many languages side
by side) are being proposed, and eventually there will unquestionably
be a shift away from ASCII to a new multilanguage standard\footnote{
        This shift will break many, many existing programs.  Converting
        all of these programs will keep many, many programmers busy
        for some time.}.

%% This section isn't really very good, and
%% is unnecessary for CS50 anyway, so out it goes...
%% \input{datarep/datarep-representing-rational-numbers-de1}

\section{Representing Programs}

Just as sequences of bits can be used to represent numbers, they can
also be used to represent instructions for a computer to perform. 
Unlike the two's complement notation for integers, which is a standard
representation used by nearly all computers, the representation of
instructions, and even the set of instructions, varies widely from one
type of computer to another.

The {\sc Ant-8} architecture, which is the focus of the rest of this
document, uses a relatively simple and straightforward representation. 
Each instruction is exactly 16 bits in length, and consists of several
bit fields, as depicted in
figure~\ref{datarep-ant-instruction-figure}.

\begin{figure}[hbtp]
\caption{{\sc Ant-8} Instruction Formats}
\label{datarep-ant-instruction-figure}
\begin{center}
\begin{tabular}{|p{1.0in}|p{1.0in}|p{1.0in}|p{1.0in}|}
\hline
4 bits & 4 bits & 4 bits & 4 bits \\
\hline
\hline
op & des & reg1 & reg2 \\
\hline
op & des & reg1 & 4-bit constant \\
\hline
op & reg & \multicolumn{2}{|l|}{8-bit constant} \\
\hline
\end{tabular}
\end{center}
\end{figure}

The first four bits (reading from the left, or high-order bits) of
each instruction are called the {\tt op} field.  The op field
determines what operation the instruction represents.  Depending on
what the op is, the rest of the instruction may represent the names of
registers or constants used by the op.

For example, the instruction $1234_{16}$ has an op of 1, which
corresponds to the operation of loading a constant ({\tt
lc}).\footnote{The fact that most of the instructions consist of four
4-bit fields makes hexadecimal notation particularly appropriate for
expressing {\sc Ant-8} instructions.} With the addition operation, the
next 4-bit fields is interpreted as the names of the register to use
load the constant into, and the last 8 bits are the constant to load. 
Therefore, instruction $1234_{16}$ places the value {\tt 0x34} into
register 2.  (The {\tt lc} instruction and the rest of the {\sc Ant-8}
instructions are described more fully in the {\sc Ant-8} tutorial.)

\section{Memory Organization}

We've seen how sequences of binary digits can be used
to represent numbers, characters, and instructions.
In a computer, these binary digits
are organized and manipulated in discrete groups,
and these groups are said to be the {\em memory} of the computer.

\subsection{Units of Memory}

The smallest of these groups, on most computers,
is called a {\em byte}\index{byte}.
On nearly all currently popular computers a byte is composed of 8 bits.

The next largest unit of memory is usually composed of
16 bits.  What this unit is called varies from computer
to computer-- on smaller machines, this is often called
a {\em word}\index{word}, while on newer architectures that
can handle larger chunks of data,
this is called a {\em halfword}\index{halfword}.

The next largest unit of memory is usually composed of 32 bits.
Once again, the name of this unit varies-- on smaller machines,
it is referred to as a {\em long}\index{long}, while
on newer and larger machines it is called a {\em word}\index{word}.

Finally, on the newest machines, the computer also can handle
data in groups of 64 bits.  On a smaller machine, this is known
as a {\em quadword}\index{quadword}, while on a larger machine
this is known as a {\em long}\index{long}.

\subsubsection{Historical Perspective}

There have been architectures that have used nearly
every imaginable word size-- from 6-bit bytes to 9-bit bytes,
and word sizes ranging from 12 bits to 48 bits.
There are even a few architectures that have no fixed
word size at all (such as the CM-2) or word sizes that
can be specified by the operating system at runtime.

Over the years, however, most architectures have converged
on 8-bit bytes and 32-bit longwords.
An 8-bit byte is a good match for the ASCII character set
(which has some popular extensions that require 8 bits),
and a 32-bit word has been, at least until recently,
large enough for most practical purposes.

\subsection{Addresses and Pointers}

Each unique byte\footnote{
	In some computers, the smallest distinct unit of memory
	is not a byte.  For the sake of simplicity, however,
	this section assumes that the smallest distinct unit
	of memory on the computer in question is a byte.}
of the computer's memory is given a unique identifier, known
as its {\em address}.  The {\em address of} a piece of memory is
often referred to as a {\em pointer to} that piece of memory--
the two terms are synonymous, although there are many contexts
where one is commonly used and the other is not.

The memory of the computer itself is often organized as a large array
(or group of arrays) of bytes of memory.  In this organization, the
address of each byte of memory is simply the index of the memory array
location where that byte is stored.


\subsection{Summary}

In this chapter, we've seen how computers represent integers
using groups of bits, and how basic arithmetic and other
operations can be performed using this representation.

We've also seen how the integers or groups of bits can be used
to represent several different kinds of data,
including written characters (using the ASCII character codes),
instructions for the computer to execute,
and addresses or pointers, which can be used to reference other data.

There are also many other ways that information can be represented
using groups of bits, including representations for rational
numbers (usually by a representation called {\em floating point}),
irrational numbers, graphics, arbitrary character sets, and
so on.  These topics, unfortunately, are beyond the scope of this
chapter.


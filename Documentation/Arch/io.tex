% $Id: io.tex,v 1.2 2002/04/16 01:02:40 ellard Exp $


\OneRegisterOp{io}{Input/Output}{\OPCODE}{reg}{const8}

\index{io@{\tt in} for Ant-8}
\index{in@{\tt in} for Ant-8}
\index{out@{\tt out} for Ant-8}

The {\tt io} instruction is used to perform either or input or output
of a single character to or from a peripheral device.

The upper 4 bits of the {\em const8} determine the direction of I/O:

\begin{itemize}

\item If the upper 4 bits are {\tt 0000}, then the byte is read from
	the peripheral into the specified register.

\item If the upper 4 bits are {\tt 0001}, then the value in {\em reg}
	is written to the peripheral.

\item The results of this instruction are undefined for any other
	values of the top four bits of {\em reg}.

\end{itemize}

The lower 4 bits of the {\em const8} determine the ``peripheral'' that
the byte is read or written to.  The standard Ant-8 implementation has three 
peripherals:

\begin{description}

\item[{\tt 0000}] Hexadecimal --
	the value is treated as a two-digit hexadecimal number.

\item[{\tt 0001}] Binary --
	the value is treated as a eight-digit binary number.

\item[{\tt 0002}] {\sc ASCII} --
	the value is interpreted as an ASCII character code.

\end{description}




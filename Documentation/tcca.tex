% $Id: tcca.tex,v 1.3 2002/01/02 02:13:49 ellard Exp $
%
% Trying to get into the TCCA workshop

\documentstyle[psfig,twocolumn]{article}

\setlength{\textwidth}{6.0in}
\setlength{\oddsidemargin}{0.25in}
\setlength{\evensidemargin}{0.25in}
\raggedbottom


\title{The ANT Architecture -- \\ An Architecture For CS1}
\author{Dan Ellard, Penelope Ellard, James Megquier, J. Bradley Chen}
\date{}



\begin{document}

\maketitle

% I had imagined starting with a paragraph something like
% this, to set the tone. What do you think?

% It seems a little strange that there aren't any references.
% I'm not prepared to add any right now. We might think of
% citing some stuff about MIPS to argue how complex real
% machines are.

A central goal in high-level programming languages, such as those we
use to teach introductory computer science courses, is to provide an
abstraction that hides the complexity and idiosyncrasies of computer
hardware. Although programming languages are very effective at
achieving this goal, certain properties of computer hardware cannot be
hidden, or are useful to know about.
As a consequence, many of the greatest conceptual challenges
for beginning programmers arise from a lack of understanding of the basic
properties of the hardware upon which computer programs execute.

To address this problem, we have developed a simple virtual machine
called ANT for use in our introductory computer science (CS1)
curriculum. ANT is designed to be simple enough that a CS1 student can
quickly understand it, while at the same time providing an accurate model of
many important properties of computer hardware. After two years of
experience with ANT in our CS1 course, we believe it is a valuable
tool for helping young students understand how programs and data are
actually represented in a computer system.

This paper gives a short introduction to ANT. We start with a brief
description of the architecture, and then describe how we use ANT in our
CS1 course. We include specific examples that demonstrate how ANT can
give students intuition about pointers, the representation of data in
memory, and other key concepts.

\section{A Description of ANT}

ANT is a simple and extremely small architecture.  The guiding
philosophy of the ANT architecture is simplicity, but not at the cost
of functionality.  The result is an architecture that is simple in
every important aspect:  easy for students to learn and to implement
as a virtual machine, and with a machine language that is easy to
assemble.  Despite this simplicity, the ANT architecture is rich
enough to support many simple but interesting applications, and to
accurately model how computers actually execute their programs.
   
The ANT instruction set follows the RISC design philosophy.  It uses
fixed-width instructions with opcodes, registers, and constants
in fixed positions so that the instructions are very easy to decode.
The first four bits of every instruction are the opcode.
The second four bits always represent a register number.
The final eight bits represent either a pair of register
numbers, a register number and a 4-bit constant, or an 8-bit constant.
Figure \ref{inst-label} describes the ANT instruction formats.
Note that the use of four-bit fields means
that an ANT instruction can be trivially written or read in hexadecimal.

\begin{figure*}
\hrule
\caption{\label{inst-label} The ANT Instruction Formats}
\vspace{3mm}

{\small
All fields are four bits wide, unless otherwise noted.

\begin{itemize}

\item The instruction format for register-to-register instructions.

	\vspace{3mm}
	\centerline{\hbox{
	\psfig{height=12mm,angle=270,figure=Figs/ant-inst3.eps}}}

\item The {\tt lc} (load constant), {\tt inc} (increment),
	{\tt jmp} (jump) and {\tt sys}
	(system call) instructions use a variant of the standard
	format that allows for an 8-bit constant.
 
	\vspace{3mm}
	\centerline{\hbox{
	\psfig{height=12mm,angle=270,figure=Figs/ant-inst8.eps}}}

\item The {\tt ld} (load) and {\tt st} (store) instructions use a four-bit
	constant in place of the third register specifier.

	\vspace{3mm}
	\centerline{\hbox{
	\psfig{height=12mm,angle=270,figure=Figs/ant-inst4.eps}}}

\end{itemize}
\vspace{3mm}
}
\hrule
\end{figure*}

Also following the RISC philosophy, the only ANT instructions that
directly access memory are the load and store instructions.

The ANT register set consists of 14 general-purpose registers and two
special-purpose registers.  The first special-purpose register, {\tt
r0}, always contains the constant zero, and the second, {\tt r1}, is
used to hold results related to the previous instruction (such as
overflow or underflow from arithmetic operations).  There
are no status registers or condition codes, and the program
counter is not directly accessible to the programmer.

ANT uses 8-bit two's-complement integers, 8-bit addresses, and 16-bit
instruction words.  In its current implementation, it uses separate
address spaces for instructions and data, making it possible to have
as many as 256 16-bit instructions and 256 8-bit words of data
simultaneously.  These address spaces could be combined, but this
would crowd the tiny address space even further, and introduce the
possibility of new kinds of programming bugs, such as programs that
accidently overwrite their own instructions.

The ANT architecture also includes a single {\tt sys} instruction,
which can perform a variety of tasks, depending on its parameters:  it
can be used to halt the processor, dump the contents of registers and
memory (for debugging), or to read or write characters or integers from or to
standard I/O.

A block diagram of the ANT architecture is shown in Figure \ref{chip-fig}.

\begin{figure*}
\caption{\label{chip-fig} A block diagram of the ANT architecture. }
\vspace{5mm}
\centerline{\hbox{
\psfig{height=3.0in,figure=Figs/ant-chip2.eps}}}
\vspace{3mm}
\end{figure*}

There are currently only twelve instructions in the ANT architecture,
including the {\tt sys} instruction.  Although this seems like a very
small number of instructions, we have found this set to be entirely adequate
for our use of ANT in our introductory course-- we have even
considered {\em removing} some infrequently-used instructions for
future versions of ANT.

Despite its small number of instructions and tiny address
space, the ANT architecture is full-featured enough to support a wide
variety of interesting programs of a level of complexity similar to
the first several programming assignments in a CS1 course.  Some
examples include:

\begin{itemize}

\item {\tt hello.asm} - The ``Hello World'' program, written in ANT
	assembly language. Uses three instructions.
	This program is shown in Figure \ref{hello-asm}.

\item {\tt echo.asm} - copy stdin to stdout one character at a time. 
	Uses seven instructions.

\item {\tt reverse.asm} - reads lines from the user, and prints them
	out in reversed order.  Uses 27 instructions.
	This program is shown in Figure \ref{reverse-asm}.

\item {\tt sort\_example.asm} - bubble-sort numbers read from user. 
	Uses 40 instructions.

\item {\tt rotate.asm} - prints ``rotated'' versions of a string.  Uses
	52 instructions.

\item {\tt hi-q.asm} - plays the game of Hi-Q.  Uses 253 instructions.
	The students write a Hi-Q game in C as part of an earlier
	assignment, so they can easily understand this program.

\end{itemize}

In fact, it is possible to implement a slightly simplified ANT virtual
machine in ANT itself, and execute simple ANT programs on this virtual
machine.

We typically assign short assembly language programs, such as
{\tt reverse.asm}, as a part of student assignments, but we do not assign
longer assembly language programs.  We do not expect our students to
become seasoned or even particularly proficient
assembly language programmers; our
goal of giving them some intuition about architecture is achieved well
before that point.  Nonetheless, ANT programs such as {\tt rotate} and
{\tt hi-q}
provide students with examples of assembly language programming, and
demonstrate that interesting programs can be implemented for the
ANT virtual machine.  This gives motivated students an opportunity to
exercise their creativity and explore the design space provided by a
simple virtual machine. 

The limited size of the ANT address space makes it very easy to
understand and debug ANT programs-- in fact, the ANT debugger can
display the contents of all of the registers and the {\em entire}
contents of data memory in a single 24-by-80 text window.  This
simplicity is a crucial feature that distinguishes ANT from a ``real''
architecture.  The smallness of the machine means that programs never
get too big or complex.  As a result, there are relatively few bugs
that our students cannot find and fix themselves, which increases
their self-confidence and reduces the time required from teaching
staff.

Although ANT is fully adequate for achieving our educational goals,
there are many things that were purposefully left out of the architecture,
in order to keep it small and simple.  ANT
contains no support for interrupts, memory protection, any kind of
MMU, or any of the other features that would be required to implement a
full-fledged operating system or run more than one program at a time. 
ANT also does not include a number of architectural features that are
common in modern processors-- it does not include any sort of cache,
nor does it have an execution pipeline.  It does not include support
for floating point arithmetic, and supports only a limited set of I/O
operations. 
Although it has some instructions that can be used to implement a
stack, there is no hardware support for a stack pointer.
(We do not ask students to write ANT programs that would require
a stack.)
These features would be helpful
for writing more complex ANT programs, or to provide a more realistic
example of a computer architecture-- but detailed realism is not our goal.
Even so, many of these features are mentioned in discussions of how we
could make the architecture more powerful and realistic.

An extension we have considered is increasing the data word size to
16 bits.  In considering such features we weigh the advantages of
enabling new applications against the risk of degrading the simplicity
and elegance of the architecture, and generally give priority to the
latter.
Because the complete machine specification for ANT is less than seven
pages, students can be expected to understand every detail of the
specification.  We consider this to be essential, and we believe that
this would be impossible for any real architecture.

\section{ANT Programming}

The ANT programming environment consists of an assembler, interpreter,
and debugger.  The assembler converts an assembly language source file
into a simplified format that the interpreter can directly read and
execute.

The debugger is an extended version of the interpreter, with debugging
functionality added.  It allows the user to set and remove
breakpoints, generate a program trace, single step through a program,
or reinitialize the processor, as well as examine the contents of
registers and data memory and disassemble the instructions.

It would be easy to combine the assembler, interpreter, and debugger
into one program.  We did not choose this approach, however, because
it would make some of our assignments more difficult-- this year, we
had students write both an ANT interpreter and an ANT assembler, and
we believe that keeping each of these programs separate helped the
students by giving them a concrete example to emulate.

\subsection{ANT Assembly Language}

This section briefly describes the instructions and system calls
available in ANT assembly language.  For a complete specification
of the assembly language, please refer to the documentation mentioned
in the conclusion of this paper.
The ANT instructions and their mnemonics are listed in
Figure \ref{arith-fig}, and the system calls are described in
Figure \ref{syscall-fig}.
The notation used in Figures \ref{arith-fig} and \ref{syscall-fig}
is documented in Figure \ref{notations}.

Note that for all instructions except {\tt sys}, register {\tt r1} is
always updated {\em after} all register reads for the instruction are
complete, so that it is always safe to use {\tt r1} as a source
register for these instructions.  (The current behavior of {\tt sys}
with respect to {\tt r1} is a relic of an early design decision, and
will be removed in the next revision of the ANT architecture.)

\begin{figure*}
{\small

\vspace{5mm}
\caption{ \label{arith-fig} The ANT Instruction Set }
\vspace{3mm}

\noindent
\begin{tabular}{|p{0.2in}|p{1.0in}|p{4.2in}|}
\hline
        {\bf Op}        & {\bf Operands}        & {\bf Description}     \\
\hline
        {\tt add}       & {\em dst, src1, src2} &
                Register {\em dst} gets {\em src1} + {\em src2}.
                {\tt r1} gets any overflow/underflow from this addition. \\
\hline
        {\tt sub}       & {\em dst, src1, src2} &
                Register {\em dst} gets {\em src1} - {\em src2}.
                {\tt r1} gets any overflow/underflow from this subtraction. \\
\hline
        {\tt mul}       & {\em dst, src1, src2} &
                Compute {\em src1} $\times$ {\em src2},
                leaving the low-order byte in register {\em dst}
                and the high-order byte in register {\tt r1}. \\
\hline
        {\tt div}       & {\em dst, src1, src2} &
                Compute {\em src1} $/$ {\em src2},
                leaving the quotient in register {\em dst}
                and the remainder in register {\tt r1}. \\
\hline
{\tt beq}       & {\em reg, src1, src2} &
        Branch to {\em reg} if ${\em src1} == {\em src2}$.
        {\tt r1} is set to the address of the instruction
        following the {\tt beq}. \\
\hline
{\tt bgt}       & {\em reg, src1, src2} &
        Branch to {\em reg} if ${\em src1} > {\em src2}$.
        {\tt r1} is set to the address of the instruction
        following the {\tt bgt}. \\
\hline
{\tt ld}        & {\em dst, src1, uconst4} & 
        Load the byte at {\em src1 + uconst4} into {\em dst}.
        {\tt r1} is unchanged.
        \\
\hline
{\tt st}        & {\em reg, src1, uconst4} &
        Store the contents of register {\em reg} to {\em src1 + uconst4}.
        {\tt r1} is unchanged.
        \\
\hline
{\tt lc}        & {\em dst, const8}      & 
        Load the constant {\em const8} into {\em dst}.
        {\tt r1} is unchanged. \\
\hline
{\tt jmp}	& {\em reg, uconst8}	&
	Branch unconditionally to the specified constant.
	{\em reg} is ignored. \\
\hline
{\tt inc}	& {\em dst, const8}	&
	Add {\em const8} to register {\em dst}. \\
\hline
        {\tt sys}   & {\em reg, code}       &
                Makes a system call.
                See Figure \ref{syscall-fig}
		for a list of the ANT system calls.
        \\
\hline
\end{tabular}

\vspace{5mm}
\caption{ \label{syscall-fig} The ANT System Calls.
Note that all syscalls set {\tt r1} to 0 if successful, and set {\tt r1} to
non-zero values to indicate failure. }
\vspace{3mm}

\noindent
\begin{tabular}{|p{0.8in}|p{0.4in}|p{4.2in}|}
\hline
{\bf Service}   & {\bf Code}    & {\bf Description} \\
\hline
{\tt halt}              & 0     & Halt the processor.
				{\em reg} is ignored.  \\
{\tt dump}              & 1     & Dump core to file {\tt ant.core}.
				{\em reg} is ignored.  \\
\hline
{\tt put\_int}           & 2     & Print the contents of {\em reg} as a
					decimal number.
                                        \\
{\tt put\_char}          & 3     & Print the contents of {\em reg} as an
                                        ASCII character.
                                        \\
{\tt put\_string}        & 4     & Print the zero-terminated ASCII string that
                                starts at {\em reg}.
                                        \\
\hline
{\tt get\_int}           & 5     &
                                Read an integer into {\em reg}.  {\em reg}
                                must not be {\tt r0} or {\tt r1}.  If
                                EOF, {\tt r1} is set to 1.  {\em Does
                                not check for illegal input.} \\
{\tt get\_char}          & 6     &
                                Read a character into {\em reg}.  {\em reg}
                                must not be {\tt r0} or {\tt r1}.  If
                                EOF, {\tt r1} is set to 1. \\
\hline
\end{tabular}

\vspace{5mm}
\caption{ \label{notations} Notations used in Figures
		\ref{arith-fig} and \ref{syscall-fig}. }
\vspace{3mm}

\noindent
\begin{tabular}{|p{0.8in}|p{4.4in}|}
\hline
{\em dst}       & Must always be a register, but never {\tt r0} or {\tt r1}. \\
\hline
{\em reg}, {\em src1}, {\em src2}       & Must always be a register. \\
\hline
{\em const8}     & Must be an 8-bit constant (-128 .. 127):
			an integer (signed), char, or label. \\
\hline
{\em uconst8}	& Must be an 8-bit constant (0 .. 255):
			an integer (unsigned) or label. \\
\hline
{\em uconst4}	& Must be a 4-bit constant integer (0 .. 15). \\
\hline
\end{tabular}

}
\end{figure*}

\subsection{ANT Assembler Directives}

\begin{itemize}

\item {\bf Comments} -
	A comment begins with a \verb$#$ and continues until the
	following newline.  The only exception to this is when the
	\verb$#$ character appears as part of an ASCII character
	constant.

\item {\bf Constants} -
	Several ANT assembly instructions contain 8-bit or 4-bit
	constants.  The 8-bit constants can be specified in a variety
	of ways:  as labels, as decimal, octal, or hexadecimal
	numbers, or as ASCII codes, using the same conventions as C. 
	The value of a label is the index of the subsequent
	instruction in instruction memory for labels that appear in
	the code, or the index of the subsequent {\tt .data} item for
	labels that appear in the data.  The 4-bit constants must be
	specified as unsigned numbers, using decimal, octal,
	or hexadecimal notation.  ASCII constants and labels
	cannot be used as 4-bit constants, even if the value
	represented fits into 4 bits.

\item {\bf The {\tt \_data\_} Label} -
	The special label {\tt \_data\_} is used to mark the boundary
	between the instructions of the program, which must appear
	before the {\tt \_data\_} label, and the data of the program,
	which must appear afterward.

\item {\bf The {\tt .data} Pseudo-Op} -
	The {\tt .data} directive is a pseudo-op that
	assembles the given bytes (8-bit integers) into the next
	available locations in the data segment.  As many as 8 bytes
	can be specified on the same line.  The byte values
	may be specified as
	hex, binary, decimal or C character constants (as described above).

	An example use of the {\tt .data} directive is shown in
	Figure \ref{hello-asm}, where {\tt .data} is used to assemble
	a string in data memory.

\end{itemize}

\begin{figure*}
\caption{ \label{hello-asm} The ``Hello World'' program, written in ANT
	assembly language. }
\vspace{3mm}
\hrule

{\small
\input{Figs/hello}
}
\hrule
\end{figure*}

\begin{figure*}
\caption{ \label{reverse-asm} The reverse program.}
\vspace{3mm}
\hrule

{\small
\input{Figs/reverse}
}
\hrule
\end{figure*}

\section{How Do We Use ANT in CS1?}

% Suggestion: make these subsections instead of a bullet-ed list.
% Give some statistics: lines of code for VM assignment,
% lines of code for asm assignment, breakdown between
% student code and code we provide.

This year, we introduced the ANT architecture in the sixth week of
CS1-- after the students have mastered the basics of C programming,
including loops, conditional execution, and arrays, but before they
are exposed to pointers, structures, or dynamic memory allocation.

Our purpose for teaching machine architecture and assembly language
programming in CS1 is to focus on basic issues of data representation
and machine architecture.  We use ANT extensively to demonstrate these
issues.  However, ANT has proven flexible enough to be tied in to a
number of other important concepts, as described below.

\subsection{Data Representation and Machine Architecture}

Our CS1 begins to introduce elements of data representation such as
binary and hexadecimal notation, two's-complement arithmetic and ASCII
codes early in the semester, followed immediately by the revelation
that computer programs themselves are stored in ``machine language''
using a very similar representation, and are executed in an entirely
mechanical manner.

We use several small ANT programs to illustrate these points, using
the ANT debugger to demonstrate how the state of the ANT machine
changes as it executes each instruction.  Finally, to reinforce these
ideas, we have students write small amounts of ANT assembly language
code themselves.

\subsection{Pointers}

Our CS1 is taught in C, and pointers in C are a subject that confuses
many CS1 students.  Over the past several years, we have experimented
with different methods of reducing the initial problems with pointers. 
Teaching a small segment on assembly language using ANT before
teaching pointers seems to help students over the initial hurdles of
understanding pointers-- once students are familiar with the
relatively simple semantics of {\tt load} and {\tt store}, the
concepts behind C pointers are much easier to grasp.  Similarly,
students who have written ANT code to stride through arrays generally
have an easier time understanding address arithmetic and the
convenient and crucial relationship of arrays and pointers in C.

\subsection{Design and Style}

It is nearly impossible to write nontrivial programs in assembly
language without careful planning and design.  Similarly, it is very
difficult to understand, modify, or debug improperly organized or
uncommented assembly language.  This is even more true for ANT
assembly language programs, because the current assembler does not
support macros or similar constructs that would make the resulting
code more readable. Therefore, the ANT assembly programming exercises
force students to think ahead before beginning to write their ANT
programs, allocating registers and choosing label names with some
care, and organizing the code in a readable manner, and it also
encourages them to carefully document their code.  We believe that
this is a pivotal experience in the course for many of our students--
particularly the students who come in to CS1 with some prior
programming experience, and therefore manage to hack through the
first few assignments without a clear design or documentation.
The ANT programming assignments are generally two short problems, each
of which can be completed in approximately one page of well-commented
code-- not dauntingly difficult, but hard enough so that the benefit
of thinking ahead is clear.

\subsection{Virtual Machines}

ANT is currently implemented only as a virtual machine, although there
has been some interest in creating a hardware implementation for use
in one of the introductory hardware architecture courses.  Since ANT
is a virtual machine, we can conveniently introduce the topic of
virtual machines in the same context as machine architecture.

To reinforce these concepts, we have students implement nearly all of
an ANT virtual machine themselves.  We supply the code to load ANT
programs from file, and code to implement most of the {\tt sys}
instruction, because at this point in the semester our students have
not seen file I/O yet.  Our students have not yet seen structures,
pointers, or dynamic memory allocation, but the assignment is designed
so that it does not require these constructs.
A well-designed solution to this assignment requires approximately 300
lines of commented C code.

For this assignment, we provided the students with an automated test
suite that they could use to test their ANT implementations for
conformance to the assignment specification.  This introduced them to
the idea of rigorous testing, and that testing could be highly
automated.  We also encouraged students to write their own tests, to add
to our test suite.

\subsection{Compilers and Assemblers}

Even after being exposed to the ideas of machine language and assembly
language, it is not clear to many students how these ideas are related
to higher-level languages.  To demystify this relationship, we ``hand
compile'' some simple C programs into ANT, leaving the students with
an intuition of how some parts of compilation can be performed.

In an assignment near the conclusion of the semester, students write a
substantial fraction of an assembler for ANT itself, allowing them to
gain deeper insights into how translation from one language to
another can be accomplished.  The students were required to implement
functions to translate
assembly language statements into machine code, check for illegal or
ill-formed instructions, maintain a symbol table of labels, and
perform backpatching.  We supply a tokenizer for the input and a
function for handling constants in hexadecimal, octal, decimal, and ASCII
because these tasks are very tedious, and the assignment was already
very long.  The solution typically required 100 lines of code for
the symbol table (not counting reuse of code from libraries
developed as part of earlier assignments) and approximately 500-700
lines of code for the rest of the assembler.

This is a substantial assignment for students at this level, and in
retrospect it was probably too difficult.  It did reinforce
the value of design and modular code, however-- students were urged to
use a modular design, and those who did found the assignment much
easier to implement and test than those who attempted an ad-hoc
design.

This assignment also utilized an automated ANT assembler validation
suite
in a manner similar to that used
in the ANT virtual machine assignment,
which was provided to the students to help them test their
assembler for conformance to the assignment specification, including
proper handling of erroneous input.

With this assignment complete, students can write their own programs
in ANT assembly language, assemble them with their own ANT assembler,
and execute them on their own ANT virtual machine.

\section{Conclusion}

We have found ANT to be very effective in our CS1 course.  It allows
introductory-level students to develop very good intuitions about such
concepts as pointers and compilation, while avoiding the complexity of
real architectures.   It also provides an effective way to discuss
topics such as virtual machines and automated testing, while the
corresponding programming assignments give ample opportunity for students
to strengthen their programming and design skills. 


% Suggestion: 

For more details on ANT, including tutorial documentation
for students, example programming exercises, and the ANT specification,
visit the ANT home page\footnote{
{\tt http://www.eecs.harvard.edu/$\sim$ellard/ANT}.}.
The ANT virtual machine, assembler,
debugger, documentation and sample ANT programs will be available
in February 1998 as a binary or source code distribution.
Please contact Dan Ellard ({\tt ellard\verb$@$eecs.harvard.edu}) for
more information.

\section{Acknowledgments}

The ANT architecture would not exist without the encouragement of
Brian Kernighan, Margo Seltzer, and the entire teaching staff of
Harvard's CS50 for 1996 and 1997.

\end{document}


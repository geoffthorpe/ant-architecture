% $Id: t32-inst.tex,v 1.11 2002/04/16 15:19:09 ellard Exp $

% Local defs
% Might eventually have to move these to a global file.

\newcommand{\rdes}{{\em rdes}}
\newcommand{\srcA}{{\em src1}}
\newcommand{\srcB}{{\em src2}}
\newcommand{\constB}{{\em const8}}
\newcommand{\constH}{{\em const16}}
\newcommand{\const}{{\em const32}}
\newcommand{\SYN}{$\bullet$}
\newcommand{\itablehead}
	{\begin{tabular}{|p{0.1in}p{0.8in}p{1.3in}|p{3.4in}|}
	\hline
	& {\bf Mnemonic}	& {\bf Operands}	& {\bf Description} \\
	}

\chapter{Ant-32 Instruction Set Summary}
\label{ant32-inst}

\section{Notation}

The notations used to describe the instructions are summarized below.

\begin{center}
\begin{tabular}{|l|p{4.0in}|}
\hline
\Reg{\em x}		& The value stored in register {\em x}.
			\\
\constB			& Any 8-bit constant.
			\\
\constH			& Any 16-bit constant.
			\\
\const			& Any 32-bit constant.  A label can be used
				as a 32-bit constant.
			\\
\SYN			& An instruction description that begins with
				a \SYN\ symbol indicates
				that the instruction is {\em
				synthetic} (see Section
				\ref{t32-inst-syn-sec}). \\

\hline
\end{tabular}
\end{center}

\section{Differences Between Assembly Language and Machine Language}
\label{t32-inst-syn-sec}

The Ant-32 assembly language is closely related to the Ant-32
machine language, and there is always a simple mapping from
instructions in the assembly language to the corresponding machine
instructions.  All of the machine language instructions are directly
expressible in assembly language, but the assembly language also
provides a slightly higher-level abstraction of the machine (called
{\em synthetic instructions}) in order to reduce the tedium of
programming in Ant-32 assembly language, and provides several directives
to the assembler 


In the tables of instructions that follow in this chapter,
instructions that begin with a \SYN\ are {\em synthetic instructions}.
Synthetic instructions fall into two categories: mnemonic names for
operations directly supported by the hardware, and names for sequences
of instructions that implement operations not directly supported by
the hardware.

For an example of the first type, consider the {\tt mov} instruction. 
The Ant-32 hardware does not implement such an instruction, but
the same functionality can be achieved by using the {\tt add}
instruction with the zero register as one of the operands.\footnote{The
{\tt mov} instruction can also be implemented in many other ways.}

\begin{verbatim}
                # copy the contents of g1 into g0
                mov     g0, g1

                # This is the same as writing:
                add     g0, g1, ze
\end{verbatim}

As an example of the second type, consider the {\tt lc} (load
constant) instruction.  The Ant-32 hardware does not implement
any method to load a 32-bit constant into a register, but the same
effect can be achieved by using an {\tt lcl} (which loads a 16-bit
constant into the lower 16 bits of a register, performing sign
extension), followed by an {\tt lch} (which loads a 16-bit constant
into the upper 16 bits of a register).

\begin{verbatim}
                # load constant 0x12345678 into g0
                lc      g0, 0x12345678

                # this is the same as writing:
                lcl     g0, 0x5678
                lch     g0, 0x1234
\end{verbatim}

In many cases, the synthetic instructions have the same form as native
instructions.  For example, {\tt addi} (add immediate) exists in the
native instruction set, but only for 8-bit constants.  The assembler
will allow {\tt addi} to take a 32-bit constant, however, by using a
synthetic sequence of instructions to implement the desired
functionality.  Note that the assembler chooses the best way to
synthesize the instruction-- for example, different sequences will be
created to implement {\tt addi} depending on whether the constant
requires 8, 16, or 32-bits to express.

\section{Loading Constants}

\itablehead

\hline
\index{lch}
	& {\tt lch} &	\rdes, \constH &
		Load \constH\ into the top (high-order) 16 bits of \rdes. \\
\index{lcl}
	& {\tt lcl} &	\rdes, \constH &
		Load \constH\ into the lower 16 bits of \rdes,
		and perform sign extension to fill in the top
		16 bits of \rdes. \\
\index{lc}
\SYN	& {\tt lc} &	\rdes, \const &
		Load the \const\ into \rdes. \\
\hline

\end{tabular}

\section{Arithmetic Operations}

\itablehead

\hline
\index{add}
	& {\tt add} &	\rdes, \srcA, \srcB	&
		\rdes\ gets \Reg{\srcA}\ + \Reg{\srcB}. \\
\index{addi}
	& {\tt addi} &	\rdes, \srcA, \constB	&
		\rdes\ gets \Reg{\srcA}\ + \constB. \\
\SYN	& {\tt addi} &	\rdes, \srcA, \const	&
		\rdes\ gets \Reg{\srcA}\ + \const. \\
\index{sub}
	& {\tt sub} &	\rdes, \srcA, \srcB	&
		\rdes\ gets \Reg{\srcA}\ - \Reg{\srcB}. \\
\index{subi}
	& {\tt subi} &	\rdes, \srcA, \constB	& \rdes\ gets \Reg{\srcA}\ - \constB. \\
\SYN	& {\tt subi} &	\rdes, \srcA, \const	& \rdes\ gets \Reg{\srcA}\ - \const. \\
\index{mul}
	& {\tt mul} &	\rdes, \srcA, \srcB	& \rdes\ gets \Reg{\srcA}\ $\times$ \Reg{\srcB}. \\
\index{muli}
	& {\tt muli} &	\rdes, \srcA, \constB 	& \rdes\ gets \Reg{\srcA}\ $\times$ \constB. \\
\SYN	& {\tt muli} &	\rdes, \srcA, \const 	& \rdes\ gets \Reg{\srcA}\ $\times$ \const. \\
\index{div}
	& {\tt div} &	\rdes, \srcA, \srcB 	& \rdes\ gets \Reg{\srcA}\ $/$ \Reg{\srcB}. \\
\index{divi}
	& {\tt divi} &	\rdes, \srcA, \constB 	& \rdes\ gets \Reg{\srcA}\ $/$ \constB. \\
\SYN	& {\tt divi} &	\rdes, \srcA, \const 	& \rdes\ gets \Reg{\srcA}\ $/$ \const. \\
\index{mod}
	& {\tt mod} &	\rdes, \srcA, \srcB 	& \rdes\ gets \Reg{\srcA}\ {\bf modulo} \Reg{\srcB}. \\
\index{modi}
	& {\tt modi} &	\rdes, \srcA, \constB 	& \rdes\ gets \Reg{\srcA}\ {\bf modulo} \constB. \\
\SYN	& {\tt modi} &	\rdes, \srcA, \const 	& \rdes\ gets \Reg{\srcA}\ {\bf modulo} \const. \\
\hline
\end{tabular}
\vspace{3mm}

The ``o'' arithmetic operations are similar to the ordinary arithmetic
operations, except that they include the calculation of the
``overflow'', if any, from the operations.  For these operations,
\rdes\ must be an even-numbered register.  The result of the operation
is stored in registers \rdes\ and \rdes\ $ + 1$.  Consult the
architecture reference for more information.

\vspace{3mm}
\noindent
\itablehead
\hline
\index{addo}
	& {\tt addo} &	\rdes, \srcA, \srcB &
		Add with overflow. \\
\index{addio}
	& {\tt addio} &	\rdes, \srcA, \constB &
		Add immediate with overflow. \\
\SYN	& {\tt addio} &	\rdes, \srcA, \const & \\
\index{subo}
	& {\tt subo} &	\rdes, \srcA, \srcB &
		Subtract with overflow. \\
\index{subio}
	& {\tt subio} &	\rdes, \srcA, \constB &
		Subtract immediate with overflow. \\
\SYN	& {\tt subio} &	\rdes, \srcA, \const & \\
\index{mulo}
	& {\tt mulo} &	\rdes, \srcA, \srcB &
		Multiply with overflow. \\
\index{mulio}
	& {\tt mulio} &	\rdes, \srcA, \constB &
		Multiply immediate with overflow. \\
\SYN	& {\tt mulio} &	\rdes, \srcA, \const & \\
\hline
\end{tabular}

\section{Logical Bit Operations}

\itablehead

\hline
\index{and}
	& {\tt and} &	\rdes, \srcA, \srcB &
		\rdes\ gets the bitwise {\sc and} of \Reg{\srcA}\ and \Reg{\srcB}. \\
\SYN \index{andi}
	& {\tt andi} &	\rdes, \srcA, \const &
		\rdes\ gets the bitwise {\sc and} of \Reg{\srcA}\ and \const. \\
\index{nor}
	& {\tt nor} &	\rdes, \srcA, \srcB &
		\rdes\ gets the bitwise {\sc nor} of \Reg{\srcA}\ and \Reg{\srcB}. \\
\SYN \index{nori}
	& {\tt nori} &	\rdes, \srcA, \const &
		\rdes\ gets the bitwise {\sc nor} of \Reg{\srcA}\ and \const. \\
\index{or}
	& {\tt or} &	\rdes, \srcA, \srcB &
		\rdes\ gets the bitwise {\sc or} of \Reg{\srcA}\ and \Reg{\srcB}. \\
\SYN \index{ori}
	& {\tt ori} &	\rdes, \srcA, \const &
		\rdes\ gets the bitwise {\sc or} of \Reg{\srcA}\ and \const. \\
\index{xor}
	& {\tt xor} &	\rdes, \srcA, \srcB &
		\rdes\ gets the bitwise {\sc xor} of \Reg{\srcA}\ and \Reg{\srcB}. \\
\SYN \index{xori}
	& {\tt xori} &	\rdes, \srcA, \const &
		\rdes\ gets the bitwise {\sc xor} of \Reg{\srcA}\ and \const. \\

\hline
\end{tabular}

\section{Bit Shifting Operations}

\itablehead

\hline
\index{shl}
	& {\tt shl} &	\rdes, \srcA, \srcB &
			Shift \Reg{\srcA}\ left by \Reg{\srcB}\ bits. \\
\index{shli}
	& {\tt shli} &	\rdes, \srcA, \constB &
			Shift \Reg{\srcA}\ left by \constB\ bits. \\
\SYN	& {\tt shli} &	\rdes, \srcA, \const &
			Shift \Reg{\srcA}\ left by \const\ bits. \\
\index{shr}
	& {\tt shr} &	\rdes, \srcA, \srcB &
			Shift \Reg{\srcA}\ right by \Reg{\srcB}\ bits. \\
\index{shru}
	& {\tt shru} &	\rdes, \srcA, \srcB &
			Unsigned shift \Reg{\srcA}\ right
			by \Reg{\srcB}\ bits. \\
\index{shri}
	& {\tt shri} &	\rdes, \srcA, \constB &
			Shift \Reg{\srcA}\ right by \constB\ bits. \\
\SYN	& {\tt shri} &	\rdes, \srcA, \const &
			Shift \Reg{\srcA}\ right by \const\ bits. \\
\index{shrui}
	& {\tt shrui} &	\rdes, \srcA, \constB &
			Unsigned shift \Reg{\srcA}\ right by \constB\ bits. \\
\SYN	& {\tt shrui} &	\rdes, \srcA, \const &
			Unsigned shift \Reg{\srcA}\ right by \const\ bits. \\
\hline
\end{tabular}
\vspace{3mm}

The left shift operation shifts the bits ``left'', towards the more
significant bits, filling in the least significant bits with zeros. 
The right shift operations shift the bits toward the least significant
bits.  If the operation is ``unsigned'' then zeros are used to fill in
the most significant bits, but if the operation is not ``unsigned''
then a copy of the most significant bit in the \srcA\ register is
used to fill these bits.

\section{Load/Store Operations}

\itablehead
\hline
\index{st1}
	& {\tt st1} &	\srcA, \srcB, \constB &
			Store the least significant byte of \Reg{\srcA}\
			to the address \Reg{\srcB} + \constB. \\
\index{st4}
	& {\tt st4} &	\srcA, \srcB, \constB &
			Store \Reg{\srcA}
			to the address \Reg{\srcB} + \constB. \\
\index{ld1}
	& {\tt ld1} &	\rdes, \srcA, \constB &
			Load the byte at address \Reg{\srcA} + \constB\
			into \rdes.   The byte is sign-extended to 32-bits. \\
\index{ld4}
	& {\tt ld4} &	\rdes, \srcA, \constB &
			Load the word at address \Reg{\srcA} + \constB\
			into \rdes.  \\
\index{ex4}
	& {\tt ex4} &	\rdes, \srcA, \constB &
			Exchange the contents of register \rdes\ and
			the word at address \Reg{\srcA} + \constB\
			\\
\hline
\end{tabular}

\section{Comparison Instructions}

\itablehead
\hline
\index{eq}
	& {\tt eq} &	\rdes, \srcA, \srcB &
		\rdes\ gets 1 if \Reg{\srcA}\ $==$ \Reg{\srcB}, 0 otherwise. \\
\index{ges}
	& {\tt ges} &	\rdes, \srcA, \srcB &
		\rdes\ gets 1 if \Reg{\srcA}\ $\ge$ \Reg{\srcB}, 0 otherwise.
		The comparison uses signed numbers. \\
\index{gts}
	& {\tt gts} &	\rdes, \srcA, \srcB &
		\rdes\ gets 1 if \Reg{\srcA}\ $>$ \Reg{\srcB}, 0 otherwise.
		The comparison uses signed numbers. \\
\index{geu}
	& {\tt geu} &	\rdes, \srcA, \srcB &
		Like {\tt ges}, but using unsigned numbers. \\
\index{gtu}
	& {\tt gtu} &	\rdes, \srcA, \srcB &
		Like {\tt gts}, but using unsigned numbers. \\
\SYN \index{les}
	& {\tt les} &	\rdes, \srcA, \srcB &
		\rdes\ gets 1 if \Reg{\srcA}\ $\le$ \Reg{\srcB}, 0 otherwise.
		The comparison uses signed numbers. \\
\SYN \index{lts}
	& {\tt lts} &	\rdes, \srcA, \srcB &
		\rdes\ gets 1 if \Reg{\srcA}\ $<$ \Reg{\srcB}, 0 otherwise.
		The comparison uses signed numbers. \\
\SYN \index{leu}
	& {\tt leu} &	\rdes, \srcA, \srcB &
		Like {\tt les}, but using unsigned numbers. \\
\SYN \index{ltu}
	& {\tt ltu} &	\rdes, \srcA, \srcB &
		Like {\tt lts}, but using unsigned numbers. \\
\hline
\end{tabular}

\section{Branch and Jump Instructions}

\itablehead
\hline
\index{jez}
	& {\tt jez} &	\rdes, \srcA, \srcB &
			If \Reg{\srcA}\ is zero,
			jump to the address \Reg{\srcB}.
			\rdes\ gets the address of the current instruction.
			\\
\index{jnz}
	& {\tt jnz} &	\rdes, \srcA, \srcB &
			If \Reg{\srcA}\ is not zero,
			jump to the address \Reg{\srcB}.
			\rdes\ gets the address of the current instruction.
			\\
\SYN \index{jezi}
	& {\tt jezi} &	\rdes, \srcA, \const &
			If \Reg{\srcA}\ is zero, jump to \const.
			\rdes\ gets the value of the current instruction.
			\\
\SYN \index{jnzi}
	& {\tt jnzi} &	\rdes, \srcA, \const &
			If \Reg{\srcA}\ is not zero, jump to \const.
			\rdes\ gets the value of the current instruction.
			\\
\index{bez}
	& {\tt bez} &	\rdes, \srcA, \srcB &
			If \Reg{\srcA}\ is zero, branch to the address of
			the current instruction plus \Reg{\srcB}.
			\rdes\ gets the address of the current instruction.
			\\
\index{bnz}
	& {\tt bnz} &	\rdes, \srcA, \srcB &
			If \Reg{\srcA}\ is not zero, branch to the address of
			the current instruction plus \Reg{\srcB}.
			\rdes\ gets the address of the current instruction.
			\\
\index{bezi}
	& {\tt bezi} &	\srcA, \constH &
			If \Reg{\srcA}\ is zero, branch to
			the address of the current instruction
			$+ ~ (4 ~ \times$ \constH$)$.
			\\
\index{bnzi}
	& {\tt bnzi} &	\srcA, \constH &
			If \Reg{\srcA}\ is not zero, branch to
			the address of the current instruction
			$+ ~ (4 ~ \times$ \constH$)$.
			\\
\hline

\end{tabular}

\section{Console I/O Instructions}

\itablehead

\hline
\index{cin}
	& {\tt cin} &	\rdes &
			Read a character from the console into \rdes. \\
\index{cout}
	& {\tt cout} &	\srcA &
			Write the character \Reg{\srcA}\ to the console. \\
\hline

\end{tabular}

\section{Halting}
\label{halt-inst-sec}

\itablehead

\hline
\index{halt}
	& {\tt halt} & 	&	Stop the processor. \\
\hline

\end{tabular}

\section{Artificial Instructions}

\itablehead

\hline
\SYN \index{mov}
	& {\tt mov} & \rdes, \srcA	& Copy \Reg{\srcA}\ to \rdes. \\
\SYN \index{j}
	& {\tt j} & \const	& Jump to \const. \\
\SYN	& {\tt j} & \srcA	& Jump to \Reg{\srcA}. \\
\SYN \index{b}
	& {\tt b} & \const	& Branch to \const. \\
\SYN	& {\tt b} & \srcA 	& Branch to \Reg{\srcA}. \\
\hline
\SYN \index{push}
	& {\tt push} & \srcA  	& Push \Reg{\srcA}\ onto the stack. \\
\SYN \index{pop}
	& {\tt pop} & \rdes 	& Pop the stack into \rdes. \\
\hline
\SYN	& {\tt call}	& \const	& Call a function
				(see Section \ref{reg32-call-sec}). \\
\SYN	& {\tt entry}	& \const	& Create a stack frame
				(see Section \ref{reg32-entry-sec}). \\
\SYN	& {\tt return}	& \srcA	& Return \Reg{\srcA}\ from a function
				(see Section \ref{reg32-return-sec}). \\
\SYN	& {\tt return}	& \const	& Return the \const\ from
				a function
				(see Section \ref{reg32-return-sec}). \\
\hline
\end{tabular}


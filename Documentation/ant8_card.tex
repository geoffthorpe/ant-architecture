% $Id: ant8_card.tex,v 1.12 2003/02/21 21:48:40 ellard Exp $

\documentclass[10pt]{report}
\setlength{\textheight}{8.0in}
\setlength{\textwidth}{6.5in}
\setlength{\oddsidemargin}{0.0in}
\setlength{\evensidemargin}{0.0in}
\raggedbottom

\pagestyle{empty}

\begin{document}

\begin{center}
{\LARGE\bf {\sc Ant-8} \input{../CurrVersion} Programming Card}
\end{center}

\vspace{3mm}
\noindent
\begin{tabular}{|ll|p{4.5in}|}
\hline
        {\bf Op}        & {\bf Operands}        & {\bf Description}     \\
\hline
\hline
        {\tt add}       & {\em des, src1, src2} &
                {\em des} gets {\em src1} + {\em src2}.
                {\tt r1} gets 1 if the result is $>$ 127,
		-1 if $<$ -128, or 0 otherwise. \\
\hline
        {\tt sub}       & {\em des, src1, src2} &
                {\em des} gets {\em src1} - {\em src2}.
                {\tt r1} gets 1 if the result is $>$ 127,
		-1 if $<$ -128, or 0 otherwise. \\
\hline
        {\tt mul}       & {\em des, src1, src2} &
                Multiply {\em src1} and {\em src2},
                leaving the low-order byte in register {\em des}
                and the high-order byte in register {\tt r1}. \\

\hline
        {\tt and}       & {\em des, src1, src2} &
                {\em des} gets the bitwise logical {\sc and} of
                {\em src1} and {\em src2}.  {\tt r1} gets the
                bitwise negation of the {\sc and} of {\em src1} and {\em src2}. \\
\hline
        {\tt nor}        & {\em des, src1, src2} &
                {\em des} gets the bitwise logical {\sc nor} of
                {\em src1} and {\em src2}.  {\tt r1} gets the
                bitwise negation of the {\sc nor}
		of {\em src1} and {\em src2}. \\
\hline
        {\tt shf}        & {\em des, src1, src2} &
		{\em des} gets the bitwise shift of {\em src1} by
		{\em src2} positions.  If {\em src2} is positive,
		{\em src1} is shifted to the left, otherwise
		{\em src1} is shifted to the right.  {\tt r1} gets 0. \\
\hline
{\tt beq}       & {\em reg, src1, src2} &
        Branch to {\em reg} if {\em src1} is equal to {\em src2}.
        {\tt r1} is unchanged. \\
\hline
{\tt bgt}       & {\em reg, src1, src2} &
        Branch to {\em reg} if {\em src1} $>$ {\em src2}.
        {\tt r1} is unchanged. \\
\hline
{\tt ld1}        & {\em des, src1, uconst4} & 
        Load the byte at {\em src1 + uconst4} into {\em des}.
	The sum of {\em} and {\em uconst4} is treated as an unsigned quantity.
        {\tt r1} is unchanged.
        \\
\hline
{\tt st1}        & {\em reg, src1, uconst4} &
        Store the contents of register {\em reg} to {\em src1 + uconst4}.
	The sum of {\em} and {\em uconst4} is treated as an unsigned quantity.
        {\tt r1} is unchanged.
        \\
\hline
{\tt lc}        & {\em des, const8}      & 
        Load the constant {\em const8} into {\em des}.
        {\tt r1} is unchanged. \\
\hline
{\tt jmp}	& {\em uconst8}	&
	Branch unconditionally to the specified constant.
	{\tt r1} is set to the address
	of the instruction following the {\tt jmp}. \\
\hline

{\tt inc}	& {\em reg, const8}	&
	Add {\em const8} to the specified register.
                {\tt r1} gets 1 if the result is $>$ 127,
		-1 if $<$ -128, or 0 otherwise. \\
\hline

{\tt in}	& {\em des, chan}	&
	Read in a byte from the peripheral specified by {\em chan},
	and place it in the destination register.  {\em chan} is
	{\tt Hex}, {\tt ASCII}, or {\tt Binary}.
	{\tt r1} gets 1 if EOI is reached, 0 otherwise. \\
\hline
{\tt out}	& {\em src1, chan}	&
	Write the byte in the {\em src1} register to the peripheral
	specified by {\em chan}. {\em chan} is
	{\tt Hex}, {\tt ASCII}, or {\tt Binary}.
	{\tt r1} gets 0. \\

\hline
        {\tt hlt}   &			&
		Dump core to {\tt ant8.core}, and halt.
		\\
\hline\hline
	{\tt .byte} & {\em byte1, $\cdots$, byteN }   &
		Assemble the given bytes (8-bit integers) into the
		next locations in the data segment.  
		8 bytes can be specified on one line.  Bytes
		may be specified as hex, octal, binary, decimal
		or C-style {\sc ASCII} constants.
                \\
\hline 
\end{tabular}

\begin{center}
\begin{tabular}{|lp{5.5in}|}
\hline
{\em des}       & Any register ({\tt r0} through {\tt r15}). The {\em des} register
			may be modified by the instruction. \\
{\em reg}       & Any register ({\tt r0} through {\tt r15}). \\
{\em src1}      & Any register ({\tt r0} through {\tt r15}). \\
{\em src2}      & Any register ({\tt r0} through {\tt r15}). \\
{\em const8}     & Any 8-bit constant (-128 .. 127):
			an integer (signed), char, or label. \\
{\em uconst8}	& Any 8-bit constant (0 .. 255):
			an integer (unsigned) or label. \\
{\em uconst4}	& Any 4-bit constant integer (0 .. 15). \\
\hline
\end{tabular}
\end{center}


\begin{description}
\item[Registers {\tt r0} and {\tt r1}]

Register {\tt r0} always contains the constant zero.  Many
instructions modify {\tt r1} as part of their operation.  Registers
{\tt r0} and {\tt r1} can be used as destination registers, but their
values are unchanged.

Registers {\tt r2} through {\tt r15} are general-purpose registers.

\item[Comments]

A comment begins with a \verb$#$ and continues until the following
newline.  The only exception to this is when the \verb$#$ character
appears as part of an ASCII character constant.

\item[Labels]

Label definitions must begin at the left margin.  All non-blank lines
that do not begin with a comment or a label definition must be
indented.  The value of the label is zero, if it begins at the start
of the program, or the address following the previous 
instruction or data value specified in the program.

Labels must begin with a letter or an underscore, followed by any
number of letters, digits, or underscores, followed by a colon.  The
colon is not part of the label name.  Label names are case-sensitive. 
A label may only be defined once in a program.  Label references must
begin with a dollar sign (\verb|$|).

\item[The \_data\_ Label]

A special label, {\tt \_data\_}, is used to mark the boundary between
the instructions of the program (which must appear before the {\tt
\_data\_} label) and the data of the program (which appear afterward).

\item[Constants]

Several {\sc Ant-8} assembly instructions contain 8-bit or 4-bit constants.

The 8-bit constants can be specified in a variety of ways:  as
decimal, octal, hexadecimal, or binary numbers, {\sc ASCII} codes
(using the same conventions as C, C++, or Java), or labels.  For
example:

\begin{center}
\begin{tabular}{|l|l|l|l|l|l|}
\hline
Decimal	& Hexadecimal	& Octal		& Binary	& {\sc ASCII} \\
\hline
{\tt 65} & {\tt 0x41} 	& {\tt 0101}	& {\tt 0b01000001} & {\tt 'A' } \\
{\tt 10} & {\tt 0xa} 	& {\tt 012}	& {\tt 0b01010} & {\tt '\verb$\$n' } \\
\hline
\end{tabular}
\end{center}

The value of a label is the index of the subsequent instruction in
instruction memory for labels that appear in the code, or the index of
the subsequent {\tt .byte} item for labels that appear in the data.

The 4-bit constants must be specified as unsigned numbers (using
decimal, octal, hexadecimal, or binary notation).  ASCII constants or
labels cannot be used as 4-bit constants, even if their value can be
represented in 4 bits.

\end{description}

\hrule
\begin{center}
{\Large
{\tt http://www.ant.harvard.edu}
}
\end{center}

\end{document}


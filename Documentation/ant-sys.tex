% $Id: ant-sys.tex,v 1.2 2002/01/02 02:13:48 ellard Exp $

\vspace{3mm}
\noindent
\begin{tabular}{|l|c|p{4.75in}|}
\hline
{\bf Service}   & {\bf Code}    & {\bf Description} \\
\hline
\hline
{\tt SysHalt}		& 0     & Halt the processor.   \\
{\tt SysDump}		& 1     & Dump core to file {\tt ant.core}.   \\
\hline
{\tt SysPutChar}	& 2     & Print the contents of {\em reg} as an
                                        ASCII character.  \\
{\tt SysGetChar}	& 3     &
                                Read a character into {\em reg}.  {\em reg}
                                must not be {\tt r0} or {\tt r1}.  If
                                EOF, {\tt r1} is set to 1. \\
\hline
{\tt SysPutInt}		& 4     & Print the contents of {\em reg} as a signed
				decimal number.
                                        \\
{\tt SysGetInt}		& 5     &
                                Read a line of input, interpret it as a
				decimal integer, and place its value
				into {\em reg}.  {\em reg}
                                must not be {\tt r0} or {\tt r1}.  If
                                EOF, {\tt r1} is set to 1.  If the line
				is too long, or empty, {\tt r1} is set to 2.
				If the line contains extraneous characters
				that are not part of the number, {\tt r1}
				is set to 3.  If the number is too large
				or too small to be represented as a signed
				8-bit number, {\tt r1} is set to 4. \\
\hline
{\tt SysPutStr}		& 6     & Print the 0-terminated {\sc ASCII}
				string that starts at {\em reg}.
				If the string is unterminated, a fault
				occurs and the ANT halts.
                                        \\
\hline
\end{tabular}


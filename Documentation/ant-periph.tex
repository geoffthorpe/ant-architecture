% $Id: ant-periph.tex,v 1.3 2002/01/02 02:13:47 ellard Exp $

\vspace{3mm}
\noindent
\begin{tabular}{|l||c|l|p{3.75in}|}
\hline
{\bf Operation}   & {\bf Format}    & {\bf Mnemonic} & {\bf Description} \\
\hline
\hline
{\tt in}	& 0     & {\tt Hex}	& Read two characters and interpret
				them as an 8-bit hexadecimal number.
				Each character must be a valid hexadecimal
				number.  \\
		& 1     & {\tt Binary}	& Read eight characters and interpret
				them as an 8-bit binary number.
				Each character must be '0' or '1'. \\
		& 2	& {\tt ASCII}	& Read a single character, as ASCII. \\
\hline
{\tt out}	& 0     & {\tt Hex}	& Write an 8-bit number as two
				characters, using hexadecimal notation.  \\
		& 1     & {\tt Binary}	& Write an 8-bit number as 8
				characters, using binary notation. \\
		& 2	& {\tt ASCII}	& Write the 8-bit number, as ASCII. \\
\hline
\end{tabular}

\vspace{3mm}

Note that the behavior of the {\tt in} operation is undefined if the
input is not properly formatted, or contains illegal characters.  For
hexadecimal input, only the characters {\tt 0-9} and {\tt A-F} (or
{\tt a-f}) may be used.  For binary input, only the {\tt 1} and {\tt
0} characters are permitted.  There is no restriction on {\sc ASCII}
characters.

\vspace{3mm}


% $Id: timer.tex,v 1.2 2002/04/16 15:33:59 ellard Exp $

\renewcommand{\INSTmnemonic}	{timer}
\renewcommand{\INSTdesc}	{Set Instruction Count Timer}
\renewcommand{\INSTopcode}	{0x4D}
\renewcommand{\INSToptype}      {ThreeReg}
\renewcommand{\INSTfieldA}	{des}
\renewcommand{\INSTfieldB}	{src}
\renewcommand{\INSTfieldC}	{\INSTunusedField}

\renewcommand{\INSTsemantic}	{

	Saves the old value of the timer to \Reg{des}, and sets the
	timer to \Reg{src}.

	If the timer is disabled, then the value in \Reg{des} will be
	zero.

	If \Reg{src} is zero or negative, then the timer is disabled
	(after its original value is saved to \Reg{des}).

	If the timer is enabled, its value is decremented every time
	an instruction is executed.  When the timer reaches zero,
	exception 20 occurs if exceptions are enabled.  If exceptions
	are disabled when the timer expires, then the exception is
	postponed until exceptions are enabled.

	When the {\tt idle} instruction is executed, if the timer is
	enabled then it is immediately decremented to zero, causing an
	immediate timer exception.

}
\renewcommand{\INSTprose}       {

	Note that this timer is really only an instruction counter and
	should not be used as a substitute for real time if accurate
	timing is necessary.

}

\renewcommand{\INSTexceptions}	{Privileged instruction, Timer}

% macros.tex
%
% Definitions of commands extending Latex for the CS51 Project Book.
%
% Bob Walton
% Dec 7, 1993

\setcounter{secnumdepth}{5}
\setcounter{tocdepth}{5}

% \AND			\wedge, the math AND operator
% \OR			\vee, the math OR operator
% \NOT			\sim, the math NOT operator
% \IMPLIES		\Rightarrow, the math IMPLIES operator
% \EQUIV		\equiv, the math EQUIVALENT-TO operator
%
% \contributors{<contributor>, <contributor>, ...}
%			Specifies comma separated list of contributors to
%			be acknowledged in the preface.
%
% \begin{history}{<filename>}
% <history>		Displays <history> as a quote if \makehistory
% \end{history}		has been included in the input.  Histories are
%			used to indicate author, date, and changes to
%			project documentation files.  Histories are not
%			printed in the copy of the project book received
%			by students.
%
%			The <filename> is the name of the file whose
%			history is being given.  It is printed at the
%			beginning of the history.  The extension is
%			assumed to be .tex, and is not included in
%			<filename>.
%
%			\end{history} must appear literally without
%			any spaces just like \end{verbatim}.
%			
% \EOL			Indicates a spot in a word where a line break
%			is OK.  E.g. {\tt car-\EOL stream}.
%
% \makehistory		Causes histories to be displayed.
%
% \inputfigure{<file>}{<height>}
%			Inputs a postscript figure from the indicated
%			file and makes it the indicated height.
%
%			Extra vertical space will be put before \inputfigure
%			only if a blank line proceeds it.  Extra vertical
%			space will be put after \inputfigure only if
%			a blank line follows it.
%
% \cbox{<code>} 	Displays <code> as code names or phrases
%			embedded in text.  Similar to \verb but
%			always terminates with }.
%
%			Restriction: \cbox cannot appear in any macro
%			argument.
%
% \$<code>$		Same as \cbox{<code>} except <code> is terminated
%			by a $.  Cannot appear in any macro argument.
%
% \dollar		The $ character.  Same as \$ in normal Latex,
%			(we redefine \$ above to allow \$<code>$).
%
% \cindex{<translated-code>}
%			Similar to index, and in fact equivalent to:
%
%			  \index{<translated-code>@\tt <translated-code>}
%
%			where <translated-code> is <code> with the special
%			characters translated for latex: namely
%				``\'' is denoted by ``$\backslash$''
%				``^'' is denoted by ``\^{~}''
%				``~'' is denoted by ``$\sim$''
%				``$'' is denoted by ``\dollar''
%				``#'' is denoted by ``\#''
%				``%'' is denoted by ``\%''
%				``&'' is denoted by ``\&''
%				``_'' is denoted by ``\_''
%				``{'' is denoted by ``\{''
%				``}'' is denoted by ``\}''
%
% \ckey{<code>}		Same as \cbox but puts <code> in index as per
%			\index.
%
%			Warning: because of limitations of latex, <code>
%			is communicated to the index as {\tt <code>},
%			and therefore <code> cannot contain special
%			characters.  If you need special characters,
%			use \ickey.
%
% \ickey{<code>}{<translated-code>}
%			Must be used in place of \ckey when <code>
%			contains characters that cannot be included
%			in the scope of \tt.
%
%			Same as \ckey but calls \cindex with
%			<translated-code> instead of <code>.
%
% \key{<text>}		Applies \em or \bf to <text> and also puts <text>
%			in the index.
%
% \ikey{<text>}{<itext>}
%			Applies \em or \bf to the first argument <text>,
%			and puts the second argument <itext> in the index.
%
% \begin{code}{<name>}	Displays <code>.  Similar to \begin{verbatim}
% <code>		but indents or centers <code>.
% \end{code}
%			\end{code} must appear literally without
%			any spaces just like \end{verbatim}.
%
%			<name> is ignored during text processing.  It is
%			used to extract the <code> from the file so it can
%			be tested.
%
%			Extra vertical space will be put before \begin{code}
%			only if a blank line proceeds it.  Extra vertical
%			space will be put after \end{code} only if
%			a blank line follows it.
%
% \begin{code*}{<name>}	Ditto but displays spaces as square cups, like
% <code>		verbatim*.
% \end{code*}		WARNING: this macro has NEVER WORKED.
%
% \begin{labpar}{<label>}
%			Begin labeled paragraph, which is an indented
%			paragraph with <label> in the left margin.
% \end{labpar}		End labeled paragraph.
%
% \begin{indentpar}	Begin indented paragraph ({list} with no label).
% \end{indentpar}	End indented paragraph.
%
% \spc			A temporary length command usable in conjunction
%			with \settowidth.
%
% \begin{problems}	Begin a list of problems.
% \problem		Start the next problem.
% \end{problems}	End a list of problems.
%
% \begin{moreproblems}	Continue a list of problems.
%			(Just like problems environment,
%			 but does not reset problem number counter.)
% \end{moreproblems}	End a continued list of problems.
%
% \begin{subproblems}	Begin a list of subproblems.
% \subproblem		Start the next subproblem.
% \end{subproblems}	End a list of subproblems.
%
% \begin{boxpicture}{<x-size>}{<y-size>}
% <picture commands>	Like \begin{picture} put sets the size of one
% \end{boxpicture}	unit so that 100 units is the side of a square
%			box suitable from making dotted pair diagrams.
%			<x-size> and <y-size> are the sizes of the
%			picture in these units.  (0,0) is the lower
%			left corner, (<x-size>-1,<y-size>-1) is the
%			upper right corner.
%
% \cons{<x>}{<y>}	Draws a CONS cell consisting of two side-by-side
%			boxes.  The Lower left coordinate of the left
%			box is  (<x>,<y>), while the lower left coordinate
%			of the right box is (<x>+100,<y>).
%
% \nil{<x>}{<y>}	Draws the lower left to upper right slant line
%			that indicates a box points at NIL.  <x>,<y> are
%			the lower left coordinates of the slant line.
%
% \sym{<x>}{<y>}{<name>}
%			Draws a symbol, which is just the symbol <name>,
%			a piece of text, centered in the box whose lower
%			left coordinates are <x>, <y>.
%
% \lab{<x>}{<y>}{<pos>>}{<text>}
%			Ditto, but positions the <text>:
%			    Against the left side of the box if <pos> = l.
%			    Against the right side of the box if <pos> = r.
%			    Against the top of the box if <pos> = t.
%			    Against the bottom of the box if <pos> = b.
%
%			<pos> can consist of two letters, one for
%			horizontal control and one for vertical control.
%			The default is to center.
%			
%
% \cpointer{<x1>}{<y1>}{<x2>}{<y2>}
%			Draws a pointer from a CONS cell box (either left
%			(car) or right (cdr)) to another box.  (<x1>,<y1>)
%			are the coordinates of the CONS cell box (its
%			lower leftmost point, actually), and (<y1>,<y2>)
%			are the coordinates of the target box (also
%			its lower leftmost point).  The pointer actually
%			goes from the middle of the CONS box toward the
%			middle of the target box, but stops at the
%			boundary of the target box.
%
%			Restriction: pointers must be horizontal pointing
%			right; vertical pointing down; or at 45 degree
%			angles pointing either down left of down right.
%
% \spointer{<x1>}{<y1>}{<x2>}{<y2>}
%			Ditto but the pointer, instead of starting from
%			the middle of the box, starts from the edge of the
%			box.  Used to point from a symbol to any box.
%
% \sline{<x1>}{<y1>}{<x2>}{<y2>}
%			Ditto but a line is drawn instead of a pointer
%			(i.e. there is no arrowhead).
%

\newcommand{\AND}{\wedge}
\newcommand{\OR}{\vee}
\newcommand{\NOT}{\sim}
\newcommand{\IMPLIES}{\Rightarrow}
\newcommand{\EQUIV}{\equiv}

\newcommand{\EOL}{\penalty \exhyphenpenalty}

\newcount\ATCATCODE
\ATCATCODE=\catcode`@

\catcode `@=11	% @ is now a letter

% The following are alterations of latex.tex macros.

% Variation on @ifnextchar:
\long\def\ifnexttoken#1#2#3{\let\@tempe #1\def\@tempa{#2}\def\@tempb{#3}%
	\futurelet\@tempc\@ifnch}

\def\inputfigure#1#2{\ifvmode \vspace{2ex}\else \par\fi
		    \centerline{\psfig{figure=#1,height=#2}}
		    \ifnexttoken\par{\vspace{2ex}}{}}

\begingroup \catcode `|=0 \catcode `[= 1
\catcode`]=2 \catcode `\{=12 \catcode `\}=12
\catcode`\\=12
|gdef|@xcode#1\end{code}%
	[#1|end[code]|ifnexttoken|par[|vspace[2ex]][]]
|gdef|@sxcode#1\end{code*}%
	[#1|end[code*]|ifnexttoken|par[|vspace[2ex]][]]
|long|gdef|@xhistory#1\end{history}[|end[history]]
|endgroup

\newlength{\codewidth}

\def\code #1{\ifvmode \vspace{2ex}\else \par\fi % must be before setlength
	     \setlength{\codewidth}{\linewidth}%
	     \addtolength{\codewidth}{-\parindent}%
	     \begin{minipage}[t]{\codewidth}%\small
	     \@verbatim \frenchspacing\@vobeyspaces \@xcode}
\def\endcode{\endtrivlist\if@endpe\@doendpe\fi\end{minipage}}

\@namedef{code*} #1{%
	     \ifvmode \vspace{2ex}\else \par\fi % must be before setlength
	     \setlength{\codewidth}{\linewidth}%
	     \addtolength{\codewidth}{-\parindent}%
	     \begin{minipage}[t]{\codewidth}%\small
	     \@verbatim \@xscode}
\@namedef{endcode*}{\endtrivlist\if@endpe\@doendpe\fi\end{minipage}}

\let\dollar\$
% \begingroup \catcode `|=0 \catcode `[= 1
% \catcode`]=2 \catcode `\{=12 \catcode `\}=12 \catcode `$=12
% \catcode`\\=12
% |gdef|cbox #1[|verb }#1}]
% |gdef|$[|verb $]
% |endgroup

\gdef\cindex #1{\index{#1@{\tt #1}}}
\gdef\ckey #1{\cbox{#1}\index{#1@{\tt #1}}}
\gdef\ickey #1#2{\cbox{#1}\index{#2@{\tt #2}}}

\gdef\key #1{{\em #1}\index{#1}}
\gdef\ikey #1#2{{\em #1}\index{#2}}

\def\history#1{\begingroup \@noligs \let\do\@makeother \dospecials \@xhistory}
\def\endhistory{\endgroup}
\def\makehistory{\def\history##1{\begin{quote} \small {\bf ##1}:}
		 \def\endhistory{\end{quote}}}

\def\contributors #1{\def\contributorlist{#1}}

% End of altered latex.tex macros.

\catcode `@=\ATCATCODE	% @ is now restored

\newenvironment{labpar}[1]%
	{\begin{list}{#1}{\settowidth{\leftmargin}{#1}%
			  \addtolength{\leftmargin}{\labelsep}%
			  \settowidth{\labelwidth}{#1}%
			  \setlength{\partopsep}{\parskip}%
			  \setlength{\parskip}{0in}%
			  \setlength{\topsep}{0in}%
			  \item}}%
	{\end{list}}

\newenvironment{indentpar}%
	{\begin{list}{}{}\item}%
	{\end{list}}

\newcounter{pnumber}
\newenvironment{problems}%
	{\begin{list}{\arabic{pnumber}.}{\usecounter{pnumber}}}%
	{\end{list}}
\newcommand{\problem}{\item}

\newcounter{mpnumber}
\newenvironment{moreproblems}%
	{\setcounter{mpnumber}{\value{pnumber}}%
	 \begin{list}{\arabic{pnumber}.}{\usecounter{pnumber}}%
	 \setcounter{pnumber}{\value{mpnumber}}}%
	{\end{list}}

\newcounter{spnumber}
\newenvironment{subproblems}%
	{\begin{list}{(\alph{spnumber})}{\usecounter{spnumber}}}%
	{\end{list}}
\newcommand{\subproblem}{\item}

\newlength{\spc}

\newcount\TCOUNTX
\newcount\TCOUNTY
\newcount\TCOUNTZ

\newenvironment{boxpicture}[2]%
	{\setlength{\unitlength}{0.002in}
	 \begin{picture}(#1,#2)
	}%
        {\end{picture}}

\newcommand{\cons}[2]
	{\put(#1,#2){\begin{picture}(200,100)
		     \put(0,0){\framebox(200,100){}}
		     \put(100,0){\line(0,1){100}}
		     \end{picture}
		    }
	}

\newcommand{\sym}[3]
	{\put(#1,#2){\makebox(100,100){\tt #3}}
        }

\newcommand{\lab}[4]
	{\put(#1,#2){\makebox(100,100)[#3]{\tt #4}}
        }

\newcommand{\nil}[2]
        {\put(#1,#2){\line(1,1){100}}
        }


\newcommand{\cpointer}[4]
       	{\TCOUNTX=#1
	 \advance\TCOUNTX by 50
	 \TCOUNTY=#2
	 \advance\TCOUNTY by 50

	 \ifnum#2=#4 {\TCOUNTZ=#3
		      \advance\TCOUNTZ by -\TCOUNTX
		      \put(\number\TCOUNTX,\number\TCOUNTY)
				{\vector(1,0){\TCOUNTZ}}
		     }
	\else {
		\ifnum#1<#3 {\TCOUNTZ=#3
		     	     \advance\TCOUNTZ by -\TCOUNTX
		     	     \put(\number\TCOUNTX,\number\TCOUNTY)
					{\vector(1,-1){\TCOUNTZ}}
		    	    } 
		\fi
		\ifnum#1=#3 {\TCOUNTZ=#4
			     \advance\TCOUNTZ by 100
			     \multiply\TCOUNTZ by -1
			     \advance\TCOUNTZ by \TCOUNTY
		     	     \put(\number\TCOUNTX,\number\TCOUNTY)
					{\vector(0,-1){\TCOUNTZ}}
		    	    } 
		\fi
		\ifnum#1>#3 {\TCOUNTZ=#3
			     \advance\TCOUNTZ by 100
			     \multiply\TCOUNTZ by -1
		     	     \advance\TCOUNTZ by \TCOUNTX
		     	     \put(\number\TCOUNTX,\number\TCOUNTY)
					{\vector(-1,-1){\TCOUNTZ}}
		    	    } 
		\fi
	      }

	\fi
        }

\newcommand{\spointer}[4]
       	{\TCOUNTX=#1
	 \advance\TCOUNTX by 50
	 \TCOUNTY=#2
	 \advance\TCOUNTY by 50

	 \ifnum#2=#4 {\TCOUNTZ=#3
		      \advance\TCOUNTX by 50
		      \advance\TCOUNTZ by -\TCOUNTX
		      \put(\number\TCOUNTX,\number\TCOUNTY)
				{\vector(1,0){\TCOUNTZ}}
		     }
	\else {
		\ifnum#1<#3 {\TCOUNTZ=#3
			     \advance\TCOUNTX by 50
			     \advance\TCOUNTY by -50
		     	     \advance\TCOUNTZ by -\TCOUNTX
		     	     \put(\number\TCOUNTX,\number\TCOUNTY)
					{\vector(1,-1){\TCOUNTZ}}
		    	    } 
		\fi
		\ifnum#1=#3 {\TCOUNTZ=#4
			     \advance\TCOUNTY by -50
			     \advance\TCOUNTZ by 100
			     \multiply\TCOUNTZ by -1
			     \advance\TCOUNTZ by \TCOUNTY
		     	     \put(\number\TCOUNTX,\number\TCOUNTY)
					{\vector(0,-1){\TCOUNTZ}}
		    	    } 
		\fi
		\ifnum#1>#3 {\TCOUNTZ=#3
			     \advance\TCOUNTX by -50
			     \advance\TCOUNTY by -50
			     \advance\TCOUNTZ by 100
			     \multiply\TCOUNTZ by -1
		     	     \advance\TCOUNTZ by \TCOUNTX
		     	     \put(\number\TCOUNTX,\number\TCOUNTY)
					{\vector(-1,-1){\TCOUNTZ}}
		    	    } 
		\fi
	      }

	\fi
        }

\newcommand{\sline}[4]
       	{\TCOUNTX=#1
	 \advance\TCOUNTX by 50
	 \TCOUNTY=#2
	 \advance\TCOUNTY by 50

	 \ifnum#2=#4 {\TCOUNTZ=#3
		      \advance\TCOUNTX by 50
		      \advance\TCOUNTZ by -\TCOUNTX
		      \put(\number\TCOUNTX,\number\TCOUNTY)
				{\line(1,0){\TCOUNTZ}}
		     }
	\else {
		\ifnum#1<#3 {\TCOUNTZ=#3
			     \advance\TCOUNTX by 50
			     \advance\TCOUNTY by -50
		     	     \advance\TCOUNTZ by -\TCOUNTX
		     	     \put(\number\TCOUNTX,\number\TCOUNTY)
					{\line(1,-1){\TCOUNTZ}}
		    	    } 
		\fi
		\ifnum#1=#3 {\TCOUNTZ=#4
			     \advance\TCOUNTY by -50
			     \advance\TCOUNTZ by 100
			     \multiply\TCOUNTZ by -1
			     \advance\TCOUNTZ by \TCOUNTY
		     	     \put(\number\TCOUNTX,\number\TCOUNTY)
					{\line(0,-1){\TCOUNTZ}}
		    	    } 
		\fi
		\ifnum#1>#3 {\TCOUNTZ=#3
			     \advance\TCOUNTX by -50
			     \advance\TCOUNTY by -50
			     \advance\TCOUNTZ by 100
			     \multiply\TCOUNTZ by -1
		     	     \advance\TCOUNTZ by \TCOUNTX
		     	     \put(\number\TCOUNTX,\number\TCOUNTY)
					{\line(-1,-1){\TCOUNTZ}}
		    	    } 
		\fi
	      }

	\fi
        }

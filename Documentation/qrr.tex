% Dan Ellard -- 02/13/94
% $Id: qrr.tex,v 1.10 2003/02/14 16:52:04 ellard Exp $
%

\section{{\sc Ant-8} Architecture Overview}

The {\sc Ant-8} architecture is a load/store architecture; the only
instructions that can access memory are the {\em load} and {\em store}
instructions.  All other operations access only registers (or, in the
case of {\tt in} and {\tt out}, access peripherals).

The {\sc Ant-8} CPU has 16 registers, named {\tt r0} through {\tt r15}.
Register {\tt r0} always contains the constant 0, and register
{\tt r1} is used to hold results related to previous operations
(described later).  {\tt r0} and {\tt r1} are {\em read-only}.
They can be used as destination registers, but their values are
unchanged.
The other 14 registers ({\tt r2} through {\tt r15}) are
general-purpose registers.

\section{Instructions}
\label{mnemonic-sec}

In the description of the instructions, the notation described in
the following table is used:

\vspace{3mm}
\begin{center}
\begin{tabular}{|lp{5.5in}|}
\hline
{\em des}       & Must always be a register. The {\em des} register
			may be modified by the instruction. \\
{\em reg}       & Must always be a register. \\
{\em src1}      & Must always be a register. \\
{\em src2}      & Must always be a register. \\
{\em const8}     & Must be an 8-bit constant (-128 .. 127):
			an integer (signed), char, or label. \\
{\em uconst8}	& Must be an 8-bit constant (0 .. 255):
			an integer (unsigned) or label. \\
{\em uconst4}	& Must be a 4-bit constant integer (0 .. 15). \\
\hline
\end{tabular}
\end{center}
\vspace{3mm}

The {\sc Ant-8} assembly language instructions are listed in Figure
\ref{mnemonic-table}.

Note that for all instructions, register {\tt r1} is always
updated {\em after} the rest of the instruction is done,
so that it is always safe to use {\tt r1} as a source register.

\begin{figure}[htp]
\caption{ \label{mnemonic-table} {\sc Ant-8} Instruction Mnemonics }
\vspace{3mm}
\noindent
\begin{tabular}{|ll|p{4.5in}|}
\hline
        {\bf Op}        & {\bf Operands}        & {\bf Description}     \\
\hline
\hline
        {\tt add}       & {\em des, src1, src2} &
                {\em des} gets {\em src1} + {\em src2}.
                {\tt r1} gets 1 if the result is $>$ 127,
		-1 if $<$ -128, or 0 otherwise. \\
\hline
        {\tt sub}       & {\em des, src1, src2} &
                {\em des} gets {\em src1} - {\em src2}.
                {\tt r1} gets 1 if the result is $>$ 127,
		-1 if $<$ -128, or 0 otherwise. \\
\hline
        {\tt mul}       & {\em des, src1, src2} &
                Multiply {\em src1} and {\em src2},
                leaving the low-order byte in register {\em des}
                and the high-order byte in register {\tt r1}. \\

\hline
        {\tt and}       & {\em des, src1, src2} &
                {\em des} gets the bitwise logical {\sc and} of
                {\em src1} and {\em src2}.  {\tt r1} gets the
                bitwise negation of the {\sc and} of {\em src1} and {\em src2}. \\
\hline
        {\tt nor}        & {\em des, src1, src2} &
                {\em des} gets the bitwise logical {\sc nor} of
                {\em src1} and {\em src2}.  {\tt r1} gets the
                bitwise negation of the {\sc nor}
		of {\em src1} and {\em src2}. \\
\hline
        {\tt shf}        & {\em des, src1, src2} &
		{\em des} gets the bitwise shift of {\em src1} by
		{\em src2} positions.  If {\em src2} is positive,
		{\em src1} is shifted to the left, if {\em src2}
		is negative {\em src1} is shifted to the right. \\
\hline
{\tt beq}       & {\em reg, src1, src2} &
        Branch to {\em reg} if {\em src1} is equal to {\em src2}.
        {\tt r1} is unchanged. \\
\hline
{\tt bgt}       & {\em reg, src1, src2} &
        Branch to {\em reg} if {\em src1} $>$ {\em src2}.
        {\tt r1} is unchanged. \\
\hline
{\tt ld1}        & {\em des, src1, uconst4} & 
        Load the byte at {\em src1 + uconst4} into {\em des}.
        {\tt r1} is unchanged.
        \\
\hline
{\tt st1}        & {\em reg, src1, uconst4} &
        Store the contents of register {\em reg} to {\em src1 + uconst4}.
        {\tt r1} is unchanged.
        \\
\hline
{\tt lc}        & {\em des, const8}      & 
        Load the constant {\em const8} into {\em des}.
        {\tt r1} is unchanged. \\
\hline
{\tt jmp}	& {\em uconst8}	&
	Branch unconditionally to the specified constant.
	{\tt r1} is set to the address
	of the instruction following the {\tt jmp}. \\

\hline

{\tt inc}	& {\em reg, const8}	&
	Add {\em const8} to the specified register.
                {\tt r1} gets 1 if the result is $>$ 127,
		-1 if $<$ -128, or 0 otherwise. \\
\hline

{\tt in}	& {\em des, uconst4}	&
	Read in a byte from the peripheral specified by {\em uconst4},
	and place it in the destination register. {\tt r1} is set to 0
	if successful, 1 if EOI is reached.
	\\
\hline
{\tt out}	& {\em src1, uconst4}	&
	Write the byte in the {\em src1} register to the peripheral
	specified by {\em uconst4}. {\tt r1} is set to 0. \\

\hline
        {\tt hlt}   & 			&
		Dump core to {\tt ant8.core}, and halt. \\

\hline 
\end{tabular}
\end{figure}
\vspace{3mm}

\subsection{{\sc Ant-8} Peripherals}
\label{peripheral-sec}

The {\tt in} and {\tt out} operations set {\tt r1} to 0 if successful,
and set {\tt r1} to non-zero values to indicate failure. 

\vspace{3mm}
% $Id: ant-periph.tex,v 1.3 2002/01/02 02:13:47 ellard Exp $

\vspace{3mm}
\noindent
\begin{tabular}{|l||c|l|p{3.75in}|}
\hline
{\bf Operation}   & {\bf Format}    & {\bf Mnemonic} & {\bf Description} \\
\hline
\hline
{\tt in}	& 0     & {\tt Hex}	& Read two characters and interpret
				them as an 8-bit hexadecimal number.
				Each character must be a valid hexadecimal
				number.  \\
		& 1     & {\tt Binary}	& Read eight characters and interpret
				them as an 8-bit binary number.
				Each character must be '0' or '1'. \\
		& 2	& {\tt ASCII}	& Read a single character, as ASCII. \\
\hline
{\tt out}	& 0     & {\tt Hex}	& Write an 8-bit number as two
				characters, using hexadecimal notation.  \\
		& 1     & {\tt Binary}	& Write an 8-bit number as 8
				characters, using binary notation. \\
		& 2	& {\tt ASCII}	& Write the 8-bit number, as ASCII. \\
\hline
\end{tabular}

\vspace{3mm}

Note that the behavior of the {\tt in} operation is undefined if the
input is not properly formatted, or contains illegal characters.  For
hexadecimal input, only the characters {\tt 0-9} and {\tt A-F} (or
{\tt a-f}) may be used.  For binary input, only the {\tt 1} and {\tt
0} characters are permitted.  There is no restriction on {\sc ASCII}
characters.

\vspace{3mm}



%%% end of ant.tx
